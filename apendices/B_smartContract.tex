\section{Template for config file}{} %sem preambulo
\label{app:template}

\begin{minted}{json}
{
   "AssetId":"assetID",
   "AssetName":"assetName",
   "AssetDescription":"assetDescription",
   "Actors":[
      {
         "actorType":"type",
         "aditionalInfo":[
            {
               "key":"value"
            }
         ]
      }
   ],
   "Steps":[
      {
         "StepId":"stepID",
         "StepName":"stepName",
         "StepOrder":1,
         "ActorType":"actorType",
         "aditionalInfo":[
            {
               "key":"value"
            }
         ]
      }
   ]
}
\end{minted}

\section{Assets, asset items, steps, and actors structs}{} %sem preambulo
\label{app:structs}


\begin{minted}{go}
type Actor struct {
  ActorId    string  `json:"actorId"`
  ActorType  string  `json:"actorType"`
  ActorName  string  `json:"actorName"`
  Deleted    bool    `json:"deleted"`
  AditionalInfo map[string]string `json:"  aditionalInfo"`
}

type Step struct {
  StepId     string  `json:"stepId"`
  StepName   string  `json:"stepName"`
  StepOrder  uint    `json:"stepOrder"`
  ActorType  string  `json:"actorType"`
  Deleted    bool    `json:"deleted"`
  AditionalInfo map[string]string `json:"  aditionalInfo"`
}

type AssetItem struct {
  AssetItemId   string    `json:"assetItemId"`
  OwnerId       string    `json:"ownerId"`
  StepID        string    `json:"stepID"`
  ParentID      string    `json:"parentID"`
  Children      []string  `json:"children"`;
  ProcessDate   string    `json:"processDate"`
  DeliveryDate  string    `json:"deliveryDate"`
  OrderPrice    string    `json:"orderPrice"`
  ShippingPrice string    `json:"shippingPrice"`
  Status        string    `json:"status"`
  Quantity      string    `json:"quantity"`
  Deleted       bool      `json:"deleted"`
  AditionalInfo map[string]string `json:"  aditionalInfo"`
}

type Asset struct {
  AssetId      string       `json:"assetId"`
  AssetName    string       `json:"assetName"`
  Description  string       `json:"description"`
  AssetItems   []AssetItem  `json:"assetItems"`
  Actors       []Actor      `json:"actors"`
  Steps        []Step       `json:"steps"`
  Deleted      bool         `json:"deleted"`
  AditionalInfo map[string]string `json:"aditionalInfo"`
}
\end{minted}

\section{Main function}{} %sem preambulo
\label{app:main}

\begin{minted}{go}
func main() {
  chaincode, err := contractapi.NewChaincode(new(SmartContract))
  if err != nil {
    fmt.Printf("Error create chaincode: %s", err.Error())
    return
  }
  if err := chaincode.Start(); err != nil {
    fmt.Printf("Error starting chaincode: %s", err.Error())
  }
}
\end{minted}

\section{Create Asset}{} %sem preambulo
\label{app:CreateAsset}

\begin{minted}{go}
func (s *SmartContract) CreateAsset(
  ctx contractapi.TransactionContextInterface, assetId string,
  assetName string, description string, assetItems []AssetItem, 
  actors []Actor, steps []Step, aditionalInfo map[string]string) error {
  
  if err != nil {
    return fmt.Errorf(
      "Failed to read the data from world state: %s", err
    )
  }

  if assetJSON != nil {
    return fmt.Errorf("The asset %s already exists", assetID)
  }
  
  asset := Asset {
    AssetId:       assetId,
    AssetName:     assetName,
    Description:   description,
    AssetItems:    assetItems,
    Actors:        actors,
    Steps:         steps,
    Deleted:       false,
    aditionalInfo: aditionalInfo,
  }
  assetAsBytes, _ := json.Marshal(asset)
  if err != nil {
    return err
  }
  return ctx.GetStub().PutState("ASSET_"+assetId,assetAsBytes)
}
\end{minted}

\section{Move asset item}{} %sem preambulo
\label{app:MoveAssetItem}

\begin{minted}{go}
func (s *SmartContract) MoveAssetItem(
  ctx contractapi.TransactionContextInterface, 
  assetItemID string, newAssetItemID string, stepID string, 
  newOwnerID string, orderPrice string, shippingPrice string, 
  status string, quantity string, 
  aditionalInfo map[string]string ) error {
  
  assetItemJSON, err := s.QueryAssetItem(ctx, assetItemID)
  if err != nil {
    return err
  }
  if assetItemJSON == nil {
    return fmt.Errorf("The assetItem %s does not exists", assetItemID)
  }

  newAssetItem := AssetItem{
    AssetItemID:      newAssetItemID,
    OwnerID:          newOwnerID,
    StepID:           stepID,
    ParentID:         assetItemID,
    Children:         []string{},
    ProcessDate:      time.Now().Format("2006-01-02 15:04:05"),
    OrderPrice:       orderPrice,
    ShippingPrice:    shippingPrice,
    Status:           status,
    Quantity:         quantity,
    Deleted:          false,
    AditionalInfoMap: aditionalInfo,
  }

  assetItemAsBytes, err := json.Marshal(newAssetItem)
  if err != nil {
    return err
  }

  return ctx.GetStub().PutState(
    "ASSET_ITEM_"+newAssetItemID, assetItemAsBytes
  )
}
\end{minted}

\section{Track asset item}{} %sem preambulo
\label{app:TrackAssetItem}

\begin{minted}{go}
func (s *AssetTransferSmartContract) TrackAssetItem(
  ctx contractapi.TransactionContextInterface, 
  assetItemID string) ([]*AssetItem, error) {
  
  assetItem, err := s.QueryAssetItem(ctx, assetItemID)
  log.Print("tracking info from assetItem id: ", assetItem.AssetItemID)
  if err != nil {
    return nil, fmt.Errorf(
      "Failed to read from world state. %s", err.Error()
    )
  }

  if assetItem == nil {
    return nil, fmt.Errorf("%s does not exist", assetItemID)
  }

  trackedItems := make([]*AssetItem, 0)

  //first add the children to tracked items
  children, err := s.getChildenTree(ctx, assetItem)
  if err != nil {
    return nil, fmt.Errorf(
      "Failed to read from world state. %s", err.Error()
    )
  }

  for _, child := range children {
    fmt.Println(child)
    trackedItems = append(trackedItems, child)
  }

  //then, add the current item to tracked items
  trackedItems = append(trackedItems, assetItem)

  //finally, add the ancestor of current item to tracked items
  for {
    currentParentId, err := strconv.Atoi(assetItem.ParentID)
    if (currentParentId <= 0) {
      break
    }
    parentAssetItem, err := s.QueryAssetItem(ctx, assetItem.ParentID)
    if err != nil {
      return nil, fmt.Errorf(
        "Failed to read from world state. %s", err.Error()
      )
    }
    newParentId, err := strconv.Atoi(parentAssetItem.ParentID)
    log.Print("newParentId: ", newParentId)
    trackedItems = append(trackedItems, parentAssetItem)
    assetItem = parentAssetItem
  }
  return trackedItems, nil
}

func (s *AssetTransferSmartContract) getChildenTree(
  ctx contractapi.TransactionContextInterface, 
  assetItem *AssetItem) ([]*AssetItem, error) {
  
  tree := make([]*AssetItem, 0)
  log.Print("len(assetItem.Children): ", len(assetItem.Children))
  if len(assetItem.Children) == 0 {
    tree = append(tree, assetItem)
  } else {
    for _, childId := range assetItem.Children {
      childAssetItem, err := s.QueryAssetItem(ctx, childId)
      if err != nil {
        return nil, fmt.Errorf(
          "Failed to read from world state. %s", err.Error()
        )
      }
      if childAssetItem == nil {
        return nil, fmt.Errorf("%s does not exist", childId)
      }

      childrenTree, err := s.getChildenTree(ctx, childAssetItem)
      for _, child := range childrenTree {
        tree = append(tree, child)
      }
    }
    tree = append(tree, assetItem)
  }
  return tree, nil
}
\end{minted}

\section{Audit Methods}{} %sem preambulo
\label{app:auditMethods}

\begin{minted}{go}
func getTransactionByID(
  vledger ledger.PeerLedger, tid []byte) pb.Response {

  if tid == nil {
    return nil, fmt.Errorf("Transaction ID must not be nil.")
  }

  processedTran, err := vledger.GetTransactionByID(string(tid))
  if err != nil {
    return nil, fmt.Errorf(
      "Failed to get transaction with id %s, error %s", 
      string(tid), err.Error()
    )
  }

  bytes, err := protoutil.Marshal(processedTran)
  if err != nil {
    return nil, fmt.Errorf(err.Error())
  }

  return shim.Success(bytes)
}

func getBlockByNumber(
  vledger ledger.PeerLedger, number []byte) pb.Response {
  
  if number == nil {
    return nil, fmt.Errorf("Block number must not be nil.")
  }
  bnum, err := strconv.ParseUint(string(number), 10, 64)
  if err != nil {
    return nil, fmt.Errorf(
      "Failed to parse block number with error %s", err
    )
  }
  block, err := vledger.GetBlockByNumber(bnum)
  if err != nil {
    return nil, fmt.Errorf(
      "Failed to get block number %d, error %s", bnum, err
    )
  }
  
  bytes, err := protoutil.Marshal(block)
  if err != nil {
    return nil, fmt.Errorf(err.Error())
  }

  return shim.Success(bytes)
}

func getBlockByHash(
  vledger ledger.PeerLedger, hash []byte) pb.Response {
  
  if hash == nil {
    return nil, fmt.Errorf("Block hash must not be nil.")
  }
  block, err := vledger.GetBlockByHash(hash)
  if err != nil {
    return nil, fmt.Errorf(
      "Failed to get block hash %s, error %s", string(hash), err
    )
  }

  bytes, err := protoutil.Marshal(block)
  if err != nil {
    return nil, fmt.Errorf(err.Error())
  }

  return shim.Success(bytes)
}

func getChainInfo(vledger ledger.PeerLedger) pb.Response {
  binfo, err := vledger.GetBlockchainInfo()
  if err != nil {
    return nil, fmt.Errorf(
      "Failed to get block info with error %s", err
    )
  }
  bytes, err := protoutil.Marshal(binfo)
  if err != nil {
    return nil, fmt.Errorf(err.Error())
  }
  return shim.Success(bytes)
}

func getBlockByTxID(
  vledger ledger.PeerLedger, rawTxID []byte) pb.Response {
  txID := string(rawTxID)
  block, err := vledger.GetBlockByTxID(txID)
  if err != nil {
    return nil, fmt.Errorf(
      "Failed to get block for txID %s, error %s", txID, err
    )
  }
  bytes, err := protoutil.Marshal(block)
  if err != nil {
    return nil, fmt.Errorf(err.Error())
  }
  return shim.Success(bytes)
}

func (e *LedgerQuerier) Invoke(stub shim.ChaincodeStubInterface) pb.Response {
  args := stub.GetArgs()
  fname := string(args[0])
  cid := string(args[1])
  sp, err := stub.GetSignedProposal()
  name, err := protoutil.InvokedChaincodeName(sp.ProposalBytes)
  targetLedger := e.ledgers.GetLedger(cid)
  qscclogger.Debugf("Invoke function: %s on chain: %s", fname, cid)
  res := getACLResource(fname)

  switch fname {
  case GetTransactionByID:
    return getTransactionByID(targetLedger, args[2])
  case GetBlockByNumber:
    return getBlockByNumber(targetLedger, args[2])
  case GetBlockByHash:
    return getBlockByHash(targetLedger, args[2])
  case GetChainInfo:
    return getChainInfo(targetLedger)
  case GetBlockByTxID:
    return getBlockByTxID(targetLedger, args[2])
  }

  return nil, fmt.Errorf(
    "Requested function %s not found.", fname
  )
}
\end{minted}


\section{Backend Endpoints}{} %sem preambulo
\label{app:endpoints}

\begin{table*}[ht]
\centering
%\caption{Backend endpoints}
\label{table:endpoints}
    \begin{center}
    \begin{tabular}{|l|p{4.5cm}|p{7.664cm}|}
        %\hline 
        %\thead{Verb} & \thead{URI} & \thead{Description}\\
        \hline 
        \cellcolor{cyan}\textbf{Actors}  & \cellcolor{cyan}\textbf{URI} & \cellcolor{cyan}\textbf{Description} \\
        \hline 
        POST & /actors & creates a new actor\\
        \hline 
        GET & /actors & retrieves the actors' list\\
        \hline 
        PUT & /actors/:id & updates an existing actor\\
        \hline 
        GET & /actors/:id & retrieves an existing actor\\
        \hline 
        DELETE & /actors/:id & removes an existing actor\\
        \hline 
        \cellcolor{cyan}\textbf{Steps}  & \cellcolor{cyan}\textbf{} & \cellcolor{cyan}\textbf{} \\
        \hline 
        POST & /steps & creates a new step\\
        \hline 
        GET & /steps & retrieves the steps' list\\
        \hline 
        PUT & /steps/:id & updates an existing step\\
        \hline 
        GET & /steps/:id & retrieves an existing step\\
        \hline 
        DELETE & /steps/:id & removes an existing step\\
        \hline 
        \cellcolor{cyan}\textbf{Asset Items}  & \cellcolor{cyan}\textbf{} & \cellcolor{cyan}\textbf{} \\
        \hline 
        POST & /asset-items & creates a new asset item\\
        \hline 
        GET & /asset-items & retrieves the asset items' list\\
        \hline 
        PUT & /asset-items/:id & updates an existing asset item\\
        \hline 
        GET & /asset-items/:id & retrieves an existing asset item\\
        \hline 
        DELETE & /asset-items/:id & removes an existing asset item\\
        \hline 
        POST & /asset-items/:id & moves an asset item through the SCM\\
        \hline 
        GET & /asset-items/track/:id & tracks an asset item through the SCM\\
        \hline
        \cellcolor{cyan}\textbf{Assets}  & \cellcolor{cyan}\textbf{} & \cellcolor{cyan}\textbf{} \\
        \hline 
        POST & /assets & creates a new asset\\
        \hline 
        GET & /assets & retrieves the assets' list\\
        \hline 
        PUT & /assets/:id & updates an existing asset\\
        \hline 
        GET & /assets/:id & retrieves an existing asset\\
        \hline 
        DELETE & /assets/:id & removes an existing asset\\
        \hline 
        \cellcolor{cyan}\textbf{Audit}  & \cellcolor{cyan}\textbf{} & \cellcolor{cyan}\textbf{} \\
        \hline
         GET & /blocks/:number & Return the block specified by block number\\
        \hline 
         GET & /blocks/hash/:hash & Return the block specified by block hash\\
        \hline 
         GET & /blocks/tx-id/:tx-id & Return the transaction by Transaction ID\\
        \hline 
         GET & /transactions:id & Return the transaction by ID\\
        \hline 
         GET & /chain & Return a blockchain Info object\\
        \hline 
    \end{tabular}
    \end{center}
\end{table*}