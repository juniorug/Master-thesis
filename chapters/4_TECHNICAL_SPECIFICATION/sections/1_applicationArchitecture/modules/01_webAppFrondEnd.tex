\subsection{WebApp - FrondEnd}\label{sec:WebAppFrondEnd}
WebApp - FrontEnd is a client-server computer application that the client (including the user interface and client-side logic) runs in a web browser. This is a single-page application (SPA), a web application that interacts with the user by dynamically rewriting the current page rather than loading entire new pages from a server. This approach avoids interruption of the user experience between successive pages, making the application behave more like a desktop application.

The application is built with Angular, a JavaScript library for building user interfaces. It is a TypeScript-based open-source web application framework led by the Angular Team at Google and by a community of individuals and corporations. Used as a base in developing of single-page or mobile applications, Angular is optimal for fetching rapidly changing data that needs to be recorded. However, fetching data is only the beginning of what happens on a web page, which is why complex Angular applications usually require additional libraries for state management, routing, and interaction with an API.

The Webapp - FrontEnd is divided into two main blocks, and these are classified according to the interactions: User Interaction Modules and Backend Interactions Services.

\subsubsection{User Interaction}\label{sec:UserInteraction}
The User Interaction modules are responsible for providing web pages that will be rendered on the client's web browser. These interactions are provided by web pages grouped by the following components:

\begin{itemize}
\item Login page
\item Application configuration module
\item User handling module (actors - CRUD)
\item Data entry module (forms)
\item Data visualization module
\item Reporting module
\end{itemize}

The Login Module is responsible for display the login and authentication alternatives pages (e.g. ‘forgot my password’, ‘reset my password’). The Application Configuration module provides the features of the creation/configuration of supply chain items and supply chain flows (steps). This module is responsible for getting the information from the user to generate the configuration JSON file in the backend. The User handling module provides the features for the creation/configuration of Actors and Steps, complementing the configuration file. The Data Entry module provides form pages that allow the actors to enter data in the application, search and move asset items from a step to another. The Data Visualization module is responsible for displaying the information about asset items in the supply chain flow through steps. In the Reporting module, users can generate reports/files organized in a narrative, graphic, or tabular form, prepared on ad hoc, periodic, recurring, regular, or as required. Reports may refer to specific periods, events, occurrences, or subjects presented in written form or any other format.


\subsubsection{Backend Interaction}\label{sec:BackendInteraction}
Backend interactions happen via a service layer consisting of:

\begin{itemize}
\item Authentication service
\item Application setup service
\item User creation service (actors)
\item Data entry service (forms)
\item Data visualization service
\item Reporting service
\end{itemize}

The function of the Authentication Service is to request information from an authenticating party and validate it against the configured identity repository using the specified authentication module. After successful authentication, the user session is activated and validated across all web applications participating in an SSO environment. For example, when a user or application attempts to access a protected resource, credentials are requested by one (or more) authentication modules. Gaining access to the resource requires that the user or application be allowed based on the submitted credentials.

Application setup service provides methods to configure and edit supply chain items, and supply chain flows, defining which steps and sub-tasks will be present in this flow and which information will be present in these steps.

The User creation Service is responsible for creating users and roles to log in and use the application’s features. Only Administrators are allowed to create new users (see Actions and Actors).

Data entry service receives data from UI forms and sends them to the backend to be processed and stored.

Data visualization services provide information about the supply chain: Assets, users, and transactions, to be used by the data visualization module.

Report services generate files (Doc/PDF/XSL, etc...) from a specific period with information about the supply chain: Assets, users, and transactions.