\subsection{Data Storage}\label{sec:DataStorage}
Data storage is a general term for archiving data in electromagnetic or other forms for use by a computer or device. Different types of data storage play different roles in a computing environment. In addition to forms of hard data storage, there are now new options for remote data storage, such as cloud computing, and blockchain that can revolutionize the ways that users save and access data.  

SCM-BP uses three applications as data storages: Blockchain, Cloud filesystem and relational database better detailed on next subsections. Blockchains grow continuously because of the amount of data and code in them, which is unchanging. Therefore, an important design decision is to choose which data and calculations to keep in and out of the chain.

\subsubsection{Filesystem}\label{sec:Filesystem}
A cloud file system is a tiered storage system that provides shared access to file data. Users can create, delete, modify, read and write files, as well as logically organize them into directory trees for intuitive access.

Cloud file sharing can be defined as a service that gives multiple users simultaneous access to a cloud file data set. Cloud file sharing security is managed with user and group permissions, allowing administrators to tightly control access to shared file data.

For all file uploaded and stored in the filesystem, a locally stored digital fingerprint (hash) is saved in the blockchain, separately from the original files or content, to make it easier to confirm whether data has been altered or manipulated in a particular organization.

\subsubsection{Database}\label{sec:Database}
A relational database is a set of formally described tables from which data can be accessed or reassembled in many different ways without having to reorganize the database tables. The standard user and application programming interface (API) of a relational database is the Structured Query Language (SQL). SQL statements are used both for interactive queries for information from a relational database and for gathering data for reports.

\subsubsection{Blockchain}\label{sec:DataStorageBlockchain}

Since the blockchain consists of a Ledger and a world state database (among another components), this can also be seen as part of data storage module as it stores data. The \ref{sec:ResourceLocator} has the intelligence to decide where to search and store data with the objective of optimize storage consumption.