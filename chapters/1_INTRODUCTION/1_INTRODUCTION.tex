\xchapter{Introduction}{}

% É recomendável utilizar `\acresetall' no início de cada capítulo para reiníciar o contator de referências às siglas.
\acresetall 

Hundreds of years ago, supply chains were reasonably straightforward. Mines and farms provided natural resources to skilled craftsmen like blacksmiths and tailors, creating and selling finished products. Today's supply chains are much more complicated, fragmented, and difficult to understand. Hundreds or even thousands of supplies worldwide contribute to making and ship products purchased by a customer. Most of the time, various companies do not know each other, and a final consumer likely doesn't know anything about how, where, when, or under what condition the products passed through. This is not just a problem for consumers. Today's supply chains are so complex that even big industry players have difficulty tracking how their goods get made.

%%SUPPLY CHAIN MANAGEMENT SUBSECTION
Many systems have been developed to solve problems that come with this complexity, such as supply chain visibility and traceability. However, these systems typically store information in standard databases controlled by service providers. This centralized data storage becomes a single point of failure and risks tampering. Products' origin information may be essential, and a central server may be changed, generating doubt about confiability in that spot. As a centralized organization, it can become a vulnerable target for bribery, and then the whole system can not be trusted anymore.

The complex web of relationships that provide materials manufacture the components, assemble or mix the parts and deliver the final product to market is known as the supply chain \cite{buurman2002supply}.

Management is on the verge of a breakthrough in understanding how industrial company success depends on the interactions between information flows, materials, money, human resources, and capital equipment. The way these five flow systems interlock to amplify one another and cause change and fluctuation will form the basis for anticipating the effects of decisions, policies, organizational forms, and investment choices \cite{forrester1958industrial}.

The term \ac{SCM} has risen to prominence over the past fifteen years \cite{cooper1997supply}. For example, at the 1995 Annual Conference of the Council of Logistics Management, 13.5\% of the concurrent session titles contained the words "supply chain." At the 1997 conference, just two years later, the number of sessions containing the term rose to 22.4\%. Moreover, the term is frequently used to describe executive responsibilities in corporations \cite{la1997supply}. SCM has become such a "hot topic" that it is difficult to pick up a periodical on manufacturing, distribution, marketing, customer management, or transportation without seeing an article about SCM or SCM-related topics \cite{ross1997competing}.

La Londe and Masters proposed that a supply chain is a set of firms that pass materials forward. Usually, several independent firms are involved in manufacturing a product and placing it in the hands of the end-user in a supply chain—raw material and component producers, product assemblers, wholesalers, retailer merchants, and transportation companies are all members of a supply chain \cite{la1994emerging}. By the same token, Lambert, Stock, and Ellram define a supply chain as the alignment of firms that brings products or services to market. 

Another definition notes that a supply chain is the network of organizations involved, through upstream and downstream linkages, in the different processes and activities that produce value in the form of products and services delivered to the ultimate consumer \cite{christopher2017logistics}. In other words, a supply chain consists of multiple firms, both upstream (i.e., supply) and downstream
(i.e., distribution), and the ultimate consumer. 

As summarizing, \cite{mentzer2001defining} defines a supply chain as a set of three or more entities (organizations or individuals) directly involved in the upstream and downstream flows of products, services, finances, or information from a source to a customer. 


To begin with, the starting point of a supply chain is the extraction of raw materials and how they are first processed (preprocessed) by suppliers for delivery in the next step. The next step is called manufacturing, where the conversion process for raw materials takes place. Following this, the constructed products are passed to the distributors responsible for allocating them to multiple intermediaries, such as wholesalers and retailers. Distributors also maintain an active inventory of products, as previous products are connected to suppliers. Subsequently, wholesalers do not sell products directly to the public but other retailers, whereas retailers dispose of products purchased to end-users. Finally, consumers are who buy or receive goods or services for personal use and not for commercial resale or trade purposes \cite{litke2019blockchains}.

The manufacturer needs to validate crucial information about the natural resources they collected by reading and verifying all tags included in its transactions and then proceeding to the proper manufacturing step. Some products have more significant value by region of origin or guarantee a fair process, so traceability is essential. New transactions with new information, such as manufacturer name, field experience, and more, are sent after the internship has been completed. Then the products are delivered to distributors. Distributors can sell products to wholesalers and retailers. This process is represented by transactions that contain essential data, such as merchant and customer address, exchange value, product raw material quality, and more \cite{sauer2018extending}. 

As the distributors sell products to intermediates (generally not end-users), they must check the progress route until that stage, such as the raw material origin, manufacturer's company popularity, distributor address, and others. Retailers can audit a product's natural resource quality and get the appropriate feedback before selling it to the consumer.   After that,  when a distributor sends the product to the wholesaler, some details, such as manufacturer name, field experience, and others, are submitted after similarly completing acts. A wholesaler needs to check this information and execute their selling to another wholesaler or retailer company. The same applies to retail companies. Finally, end-users obtain the final product and should track and verify all aspects from the beginning of its supply chain journey \cite{litke2019blockchains}. 

There are billions of products being manufactured daily through complex supply chains extending to all parts of the world. However, there is very little information on how, when, and where these products originated, were manufactured, and used during their life cycle \cite{horiuchirastreabilidade}.

Before reaching the end consumer, the goods go through an often vast network of retailers, distributors, carriers, warehousing facilities, and suppliers who participate in the design, production, delivery, and sales process of a product, but in many cases. These steps are a dimension invisible to the consumer \cite{provenance2015}.

%%%% TRACEABILITY SUBSECTION
\citeonline{gryna1998juran} defines traceability as the ability to preserve the identity of the products and their origins so that the collection, documentation, and maintenance of information related to all processes in the production chain must be ensured. For a food product, traceability represents the ability to identify where and how it was grown, as well as the ability to track its post-harvest history and to identify the processes performed at each step in the production chain through records. Traceability is required primarily for \cite{horiuchirastreabilidade}: Improve credibility with customers and consumers, ensure that only quality materials and components are present in the final product, better allocate responsibilities, identify products that are distinct but may be confused, enable the return of defective or suspect products, and find the causes of failures and take steps to repair them at the lowest possible cost.

Consumers consider traceability as part of a standard protection package when purchasing products. Traceability can improve credibility in this scope since all the related providers and dealers or another agent between the raw manufacturers and the final consumer can be tracked. In a traceability system, the responsibility papers are well defined. A traceability system allows users to track products by providing information about them (e.g., originality, components, or locations) during production and distribution. Suppliers and retailers typically require independent, government-certified traceability service providers to inspect products throughout the supply chain. Vendors and retailers request traceability services for different purposes. Suppliers want to receive certificates to showcase their products. Retailers want to verify the origin and quality of products \cite{lu2017adaptable}.


In \cite{opara2003traceability} six crucial elements are to be considered for traceability:

\begin{itemize}
\item Product Traceability: Determines the location of a product at any stage of the production chain to facilitate logistics, inventory management, product recall, and information disclosure to consumers and customers.
\item Process Traceability: Identifies the type and sequence of activities that affected a particular product. This includes any interactions between the product and physical / mechanical, chemical, and environmental factors that result in the transformation of raw material into value-added products.
\item Genetic traceability: determines the genetic constitution of the product.
\item Input traceability: Determines the type and source of input, such as fertilizers and livestock.
\item Traceability of Diseases and Pests: Tracks the epidemiology of pests such as bacteria and viruses.
\item Measurement traceability: determines the measurement instruments and specifying the environmental, geospatial, and temporal factors that influence data quality.
\end{itemize}

Supply chain visibility, or traceability,  is one of the key challenges encountered in the business world. Most companies have little or no information about their own second and third-tier suppliers. Transparency and end-to-end visibility of the supply chain can help shape product, raw material, test control, and end product flow, enabling better operations and risk analysis to ensure better chain productivity \cite{abeyratne2016blockchain}.

Folinas et al. \cite{folinas2006traceability} identified that the efficiency of a traceability system depends on the ability to track and trace each asset and logistics unit in a way that enables continuous monitoring from firstly processed until final clearance by the consumer.

\citeonline{aung2014traceability} and \citeonline{golan2004traceability} set three main traceability objectives, namely: (1) better supply chain management, (2) product differentiation and quality assurance, and (3) better identification of non-compliant products. An additional objective is to maintain assurance of traceability following applicable regulations and standards. A complete traceability system will include components that manage \cite{vargas2017trazabilidad}:

\begin{enumerate}
\item Identification, marking and assignment of traceable objects, parties and locations.
\item Automatic capture (by scanning or reading) of movements or events involving an object.
\item Record and share traceability data, internally or with parts of a supply chain, to gain visibility of what has occurred.
\end{enumerate}

Traceability systems typically store information in standard databases controlled by service providers. This centralized data storage becomes a single point of failure and risks tampering. Consequently, these systems result in trust problems, such as fraud, corruption, tampering, and falsifying information. Likewise, a centralized system is vulnerable to collapse \cite{tian2017supply}.

Nowadays, a new technology called the blockchain presents a whole new approach based on decentralization. Blockchain enables end-to-end traceability, bringing a standard technology language to the supply chain while allowing consumers to access the asset's history of these products through a software application. This characteristic is provided by an immutable register ledger that facilitates recording transactions and tracking assets across a network \cite{galvez2018future}.

%%% END TRACEABILITY SUBSECTION


Blockchain and smart contracts could make supply chain management more straightforward and transparent. The idea is to create a single source of information about products and supply chains via a global ledger. Each component would have its entry on the blockchain that gets tracked over time. Both untrust companies could then update the status of goods in real-time. The result is that once the clients receive their products, they can track every piece back to its manufacturer. Theoretically, users could trace the supply chain back to the mines where the raw materials came from \cite{greve2018blockchain}.

Companies can also use the blockchain supply chain as a single source of truth for their products. They can manage and monitor risks within the supply chain, ensure the quality of delivery parts and track the delivery status. Additionally, companies can use smart contracts to manage and pay for the supply chain autonomously. This would reduce the need for large contract invoices on the back-and-forth of refund requests for faulty components. Those same smart contracts could assist with shipping and logistics tracking valuable products as they travel around the world. Using blockchain, companies can finally have a complete picture of their products at every stage in the supply chain, bringing transparency to the production process while reducing the cost of manufactured goods.

This work presents \acf{SCM-BP}, a generic framework intended to be used in any supply chain correlated to assets and products. It also presents a use case of this framework applied.

This paper is structured as follows: Chapter 2 presents several technologies involved in preparing this dissertation, introduces essential concepts of the Computer area in which the context of this project is inserted, and presents business concepts related to supply chain management. Chapter 3 shows correlated works. Chapter 4 presents the solution, its architecture, details of the executed implementation, and exposes the validation and results. The last chapter presents the conclusions and future work.