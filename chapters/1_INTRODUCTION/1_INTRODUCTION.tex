\xchapter{Introduction}{}

% É recomendável utilizar `\acresetall' no início de cada capítulo para reiníciar o contator de referências às siglas.
\acresetall 

Hundreds of years ago, supply chains were reasonably straightforward. Mines and farms provided natural resources to skilled craftsmen like blacksmiths and tailors, creating and selling finished products. Today's supply chains are much more complicated, fragmented, and difficult to understand. Hundreds or even thousands of supplies worldwide contribute to making and ship products purchased by a customer. Most of the time, various companies don't know each other, and a final consumer likely doesn't know anything about how, where, when, or under what condition the products passed through. This is not just a problem for consumers. Today's supply chains are so complex that even big industry players have difficulty tracking how their goods get made.

In order to solve problems that come with this complexity, such as supply chain visibility and traceability, many systems have been developed. However, these systems typically store information in standard databases controlled by service providers. This centralized data storage becomes a single point of failure and risks tampering. Products' origin information may be essential, and a central server may be changed, generating doubt about confiability in that spot. As a centralized organization, it can become a vulnerable target for bribery, and then the whole system can not be trusted anymore.

Blockchain and smart contracts could make supply chain management more straightforward and transparent. The idea is to create a single source of information about products and supply chains via a global ledger. Each component would have its own entry on the blockchain that gets tracked over time. Both untrust companies could then update the status of goods in real-time. The result is that once the clients receive their products, they could track every piece back to its manufacturer. Theoretically, users could trace the supply chain all the way back to the mines where the raw materials came from \cite{greve2018blockchain}.

Companies can also use the blockchain supply chain as a single source of truth for their products. They can manage and monitor risks within the supply chain, ensure the quality of delivery parts and track the delivery status. Additionally, companies can use smart contracts to manage and pay for the supply chain autonomously. This would reduce the need for large contract invoices on the back-and-forth of refund requests for faulty components. Those same smart contracts could assist with shipping and logistics tracking valuable products as they travel around the world. Using blockchain, companies can finally have a complete picture of their products at every stage in the supply chain, bringing transparency to the production process while reducing the cost of manufactured goods.

This work presents \acf{SCM-BP}, a generic framework intended to be used in any supply chain correlated to assets and products. It also presents a use case of this framework applied.

This paper is structured as follows: Chapter 2 presents several technologies involved in preparing this dissertation, introduces essential concepts of the Computer area in which the context of this project is inserted, and presents business concepts related to supply chain management. Chapter 3 shows correlated works. Chapter 4 presents the solution, its architecture, and details of the executed implementation. Chapter 5 exposes the validation and its results. The last chapter presents the conclusions and future work.