\xchapter{Conclusion and Future work}{} %sem preambulo
\acresetall 
\label{chap:conclusion}

Lately, Blockchain technology has been the subject of extensive research but rarely related to supply chain traceability. Although some companies have launched pilot projects using blockchain technology to manage their supply chains, no detailed information on the technical implementation of such projects has been reported. Either way, the retail industry has the potential to use this technology to improve traceability.

Even if some properties of blockchain implementation may be beneficial for supply chain management, there are still few uses to support this claim. With so little research on this subject, it is difficult for industry stakeholders to understand exactly how blockchain technology can be used in their specific business.

This paper has presented a framework for new decentralized traceability systems based on blockchain technology. Moreover, an example scenario was given to demonstrate how it works in an enterprise supply chain. This system delivers real-time information to all supply chain members on the safety status of goods, significantly reduces the risk of centralized information systems, and brings more secure, distributed, transparent, and collaborative to the supply chain management. The Framework can significantly improve the development time of Supply Chain Management applications and provide efficiency and transparency of products in a supply chain.

While joining a blockchain consortium benefits, all stakeholders, adopting new technology such as blockchain is challenging for traditional industries due to the learning curve and cost of integrating blockchain into existing systems. Negotiating business details also takes time. In addition, the development of smart contracts must take into account quality and adaptability. Transparency and data sharing are most important in this regard. Overall, blockchain is a good option for providing traceability in supply chain management. However, the industry needs to look more closely at its risks and opportunities.

Blockchain enables end-to-end traceability by bringing a common technological language to the supply chain while allowing consumers to access the story of goods on their labels through any connected device. This characteristic has raised the need to trace products through the complex supply chain from retail back to the farm: to trace an outbreak; to verify that a product is kosher, organic, or allergen-free; or to assure transparency to consumers. Digital product information such as farm origination details, batch numbers, factory and processing data, expiry dates, storage temperatures, and shipping details are digitally connected to items. Their information is entered into the blockchain at each step of the process. All members of the business network agree upon the information acquired in each transaction. Once consensus is reached, no permanent record can be altered. Each piece of information provides critical data that may potentially reveal safety issues with the product concerned. The record created by the blockchain can also help retailers to manage the shelf life of products in individual stores and further strengthen safeguards relating to food authenticity. Across ecosystems, business model changes enabled by blockchain technology can bring strengthened trust and transparency and a new link to value exchange. Whether it is individuals seeking to complete transactions involving many parties, or enterprises collaborating across multiple organizational silos, wherever any documents or transactions must be confirmed, settled, exchanged, signed, or validated, there are usually frictions that can be avoided by using blockchain technology to unlock greater economic value.

We propose a deeper evaluation that may analyze different product types and accomplish performance tests as future work. Besides, the role permission could be applied to guarantee that only allowed users could read/write sensitive information in the blockchain. This could be made by using a flag in the additional info to show the field as public/private information or, better, use the private data collection feature provided by Hyperledger. Also, the asset item's data structure could be changed to a tree data structure for better performance results. 