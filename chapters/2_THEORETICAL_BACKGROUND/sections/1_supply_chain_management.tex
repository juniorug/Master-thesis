\section{Supply Chain Management}\label{sec:scm}

The complex web of relationships that provide materials manufacture the components, assemble or mix the parts and deliver the final product to market is known as the supply chain \cite{buurman2002supply}.

Management is on the verge of a major breakthrough in understanding how industrial company success depends on the interactions between the flows of information, materials, money, manpower, and capital equipment. The way these five flow systems interlock to amplify one another and to cause change and fluctuation will form the basis for anticipating the effects of decisions, policies, organizational forms, and investment choices \cite{forrester1958industrial}.

The term \ac{SCM} has risen to prominence over the past fifteen years \cite{cooper1997supply}. For example, at the 1995 Annual Conference of the Council of Logistics Management, 13.5\% of the concurrent session titles contained the words "supply chain." At the 1997 conference, just two years later, the number of sessions containing the term rose to 22.4\%. Moreover, the term is frequently used to describe executive responsibilities in corporations \cite{la1997supply}. SCM has become such a "hot topic" that it is difficult to pick up a periodical on manufacturing, distribution, marketing, customer management, or transportation without seeing an article about SCM or SCM-related topics \cite{ross1997competing}.

The definition of supply chain seems to be more common across authors than the definition of "supply chain management" \cite{cooper1993characteristics, la1994emerging, lambert1998fundamentals}. La Londe and Masters proposed that a supply chain is a set of firms that pass materials forward. Normally, several independent firms are involved in manufacturing a product and placing it in the hands of the end user in a supply chain—raw material and component producers, product assemblers, wholesalers, retailer merchants and transportation companies are all members of a supply chain \cite{la1994emerging}. By the same token, Lambert, Stock, and Ellram define a supply chain as the alignment of firms that brings products or services to market. 

Another definition notes a supply chain is the network of organizations that are involved, through upstream and downstream linkages, in the different processes and activities that produce value in the form of products and services delivered to the ultimate consumer \cite{christopher2017logistics}. In other words, a supply chain consists of multiple firms, both upstream (i.e., supply) and downstream
(i.e., distribution), and the ultimate consumer. 

As a summarization, \cite{mentzer2001defining} defines a supply chain as a set of three or more entities (organizations or individuals) directly involved in the upstream and downstream flows of products, services, finances, and/or information from a source to a customer. 


To begin with, the starting point of a supply chain is the extraction of raw materials and how they are first processed (preprocessed) by suppliers for delivery in the next step. The next step is called manufacturing, where the conversion process for raw materials takes place. Following this, the constructed products are passed to the distributors who are responsible for allocating them to multiple different intermediaries, such as wholesalers and retailers. Distributors also maintain an active inventory of products, as previous products are connected to suppliers. Subsequently, wholesalers do not sell products directly to the public, but to other retailers, whereas retailers dispose products purchased to end users. Finally, Consumers are who buy or receive goods or services for personal needs or use and not for commercial resale or trade purposes \cite{litke2019blockchains}.

The manufacturer needs to validate crucial information about the natural resources they collected by reading and verifying all tags that the latter includes in its transactions and then proceeding to the proper execution of manufacturing step. New transactions with new information, such as manufacturer name, field experience and more, are sent after the internship has completed. Then the products are delivered to distributors. Distributors are able to sell products to wholesalers and retailers. This process is represented by transactions that contains important data, such as merchant and customer address, exchange value, product raw material quality, and more \cite{sauer2018extending}. 

As the distributors sell products to intermediates (generally not end users), they must check the information about the progress route until that stage, for example the raw material origin, manufactures company popularity, distributor address and others. Retailers can audit product's natural resource quality, and get the appropriate feedback before selling it to the consumer. After that, when a distributor send the product to the wholesaler, some details, such as manufacturer name, field experience and others, are submitted after the completion of acts in a similar way. Wholesaler needs to check these information and execute their selling to another wholesaler or retailer company. The same applies to the retailer companies. Finally, end users obtain the final product and should able to track and verify all aspects from the beginning of its supply chain journey \cite{litke2019blockchains}. 