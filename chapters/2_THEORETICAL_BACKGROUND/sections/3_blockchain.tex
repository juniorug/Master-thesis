\section{Blockchain}\label{sec:blockchain}
Recently, cryptocurrency has attracted extensive attention from both industry and the academy. Bitcoin, which is often called the first cryptocurrency, had a huge success with the capital market coming to \$ 10 billion in 2016 \cite{coindesk}. Blockchain is the central mechanism of the Bitcoin and was first proposed in 2008 and implemented in 2009 \cite{nakamoto2008bitcoin}. The blockchain can be considered as a public ledger, in which All committed transactions are stored in a block chain. This chain grows continuously when new blocks are attached to it \cite{zheng2016blockchain}.

At the origin of the blockchain is the Bitcoin protocol, proposed by Satoshi Nakamoto \cite{nakamoto2008bitcoin}. This article proposes a P2P network where transactions with the cryptocurrency bitcoin, proposed by customers, are received by servers, who will decide, through a consensus protocol based on cryptographic challenges, on the order in which they will be carried out and permanently stored in a chain of blocks, replicated on each server. According to \citeonline{FORMIGONI2017}, it was the creation of a digital currency that worked in a peer-to-peer (P2P) network that allowed the sending of online payments in a completely secure way, without the involvement of financial institutions, for all participants from the Web. In this sense, Blockchain was created motivated by the need for an efficient, economical, reliable and secure system to conduct and record financial transactions. Hence the question: what is the relationship between Blockchain and Bitcoin? Blockchain is the platform used for the operation of the Bitcoin network and several other cryptocurrencies.

While the system of financial institutions that serve as third parties reliable processors for processing payments work well for most still suffers from the shortcomings inherent in the model based on confidence. In addition, the cost of mediation increases transaction costs, which limits the practical minimum size of the transaction and eliminates the possibility of small occasional transactions. To solve these problems, \cite{nakamoto2008bitcoin} defined an electronic payment system called Bitcoin, based on cryptographic proof rather than reliable, allowing either party willing to transact directly with each other without the need to a reliable third party.

This revolution began with a new marginal economy on the Internet. Bitcoin emerges as an alternative currency issued and not backed by a central authority, but by automated consensus among networked users. Its true uniqueness, however, lay in the fact that it did not require that users trust each other. Through self-policing algorithmically, any malicious attempt to circumvent the system would be rejected. In a precise and technical definition, Bitcoin is a digital money. which is transacted via the Internet in a decentralized system without bail, using a ledger called blockchain. It's a new way of combining peer-to-peer file sharing rent with public key encryption \cite{swan2015blockchain}.

For \cite{swan2015blockchain}, besides the currency ( "Blockchain 1.0"), smart contracts ("2.0") demonstrate how the blockchain is in a position to become the fifth disruptive computing paradigm after mainframes, PCs, Internet and mobile/ social networks. Bitcoin is starting to become a digital currency, but technology blockchain behind it can be much more significant.

The rapid growth in blockchain technology adoption and the development of applications based on this technology have begun to revolutionize financial services industries. In addition to bitcoin, common applications of blockchain usage varies from proprietary networks used to process financial claims, insurance claims to platforms that can issue and trade equity and corporate bonds \cite{michael2018blockchain}.

 Blockchain exists with real world implementations beyond cryptocurrencies and these solutions deliver powerful benefits to healthcare organizations, bankers, retailers and consumers among others. Potential benefits of blockchain are more than just economic. They extend to the political, humanitarian, social and scientific domains. Its technological capacity is already being harnessed by specific groups to solve real world problems.



\subsection{Blockchain Properties}\label{sec:propriedades}

Blockchain technology has key features such as centralization, persistence, anonymity and auditability. Blockchain can function in a decentralized environment that is activated by the integration several key technologies such as cryptographic hash, digital signature (based on asymmetric encryption) and distributed consensus engine. With blockchain technology, a transaction may occur in a decentralized manner. As a result, blockchain can greatly save the cost and improve efficiency \cite{zheng2016blockchain}. The main properties of the blockchain are considered innovative and enable rapid adoption for technology \cite{greve2018blockchain}:

\begin{itemize}
\item Decentralization: Applications and systems run in a distributed manner, through the establishment of trust between the parties, without the need for a trusted intermediary entity. This is the main motivator for the growing interest in the blockchain
\item Availability and integrity: All datasets and transactions are replicated in different nodes in a secure way, in order to keep the system available and consistent.
\item Transparency and auditability: All transactions recorded in the ledger are public and can be verified and audited. Furthermore, technology codes are often open, verifiable.
\item Immutability and Irrefutability: Transactions recorded in the ledger are immutable. Once registered they cannot be refuted. Updates are possible based on the generation of new transactions and the realization of a new consensus.
\item Privacy and Anonymity: It is possible to offer privacy to users without the third parties involved having access and control of their data. In technology, each user manages their own keys and each server node stores only encrypted fragments of user data. Transactions are somewhat anonymous, based on the address of those involved in the blockchain.
\item Disintermediation: Blockchain enables the integration between different systems in a direct and efficient way. Thus, it is considered a connector of complex systems (systems of systems), allowing the elimination of intermediaries in order to simplify the design of systems and processes.
\item Cooperation and Incentives: Offer of an incentive-based business model, in the light of game theory. On-demand consensus is now offered as a service at different levels and scopes.
\end{itemize}

