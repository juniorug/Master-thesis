\section{Public Blockchain Versus Permissioned Blockchain}\label{sec:versus}

Blockchain networks can be categorized into two groups: public (or permissionless) blockchain and private (or federated or permissioned) blockchain (with permission and controlled access) \cite{greve2018blockchain}.

Since the beginning of blockchain technology, people have debated about public vs. permissioned blockchain. In an enterprise environment, it is essential to know the significant differences between these two. Public and permissioned blockchain examples play a considerable role in the companies looking for the perfect blockchain type for their solutions \cite{101blockchains}. Table \ref{table:pubVsPriv} presents a comparison between public and permissioned blockchain.

The primary difference between public and private blockchain is the level of access participants are granted. In order to pursue decentralization to the fullest extent, public blockchains are entirely open. Anyone can participate by adding or verifying data. The most common examples of public blockchain are Bitcoin (BTC) and Ethereum (ETH). These cryptocurrencies are created with open-source computing codes, which can be viewed and used by anyone. A public blockchain is about accessibility, which is evident in how it is used \cite{selfkeyOrg}. 

Conversely, permissioned blockchain (also known as private blockchain) only allows certain entities to participate in a closed network. Participants are granted specific rights and restrictions in the network, to have full access or limited access at the discretion of the network. As a result, permissioned blockchain is more centralized as only a small group controls the network. The most common examples of permissioned blockchains are Ripple (XRP) and Hyperledger \cite{blockgeeks2018deeper}.

Additionally, some public blockchains also allow anonymity, while permissioned ones do not. For example, anyone can buy and sell Bitcoin without having their identity revealed. It allows everyone to be treated equally. Whereas with permissioned blockchains, the identities of the participants are known. This is typical because permissioned blockchain is used in the corporate and business-to-business sphere, where it is crucial to know who is involved \cite{101blockchains}.

\subsection{Public blockchain}\label{sec:blockchainPublica}
On a public or permissionless blockchain, any person can participate without a specific identity. There are no restrictions when it comes to participation. Public blockchains typically involve a native cryptocurrency and often use \acf{PoW} consensus and economic incentives \cite{androulaki2018hyperledger}. Anyone can audit public blockchain, and each node has as much transmission power as any other. For a transaction to be considered valid, it must be authorized by all nodes constituents via the consensus process. As long as each node meets protocol-specific stipulations, their transactions can be validated and thus added to the chain \cite{Comstor2018}.

In a public blockchain, all the participants have equal rights no matter what. People can join in and participate in consensus, transact with their peers as they please. Everyone can see the ledger as well, thus maintaining transparency at all times \cite{101blockchains}.

As the P2P network node-set is unknown, its membership is dynamic, allowing random node entrances and exits and anonymity. Blockchain can act globally without the control of its participants, who do not even trust each other mutually. Are examples of public blockchain: the Bitcoin network, Ethereum, and several other cryptocurrencies \cite{bashir2018mastering, antonopoulos2017mastering}.

If a fully decentralized network system is required, then public blockchain is the way to go. However, it can get problematic when incorporating a public blockchain network with the enterprise blockchain process \cite{101blockchains}.

The main characteristics of a public blockchain are:

\begin{itemize}
\item High security: Public blockchain companies always design every single platform in a way that offers complete security.
\item Open environment: public blockchain is open for all. The system typically has an incentive mechanism to encourage more participants to join the network.
\item Anonymous nature: Everyone is anonymous. Users will not use their real name or real identity here. Everything would stay hidden, and no one can track data based on that.
\item No regulations: Public blockchain does not have any regulations that the nodes have to follow. So, there is no limit to how one can use this platform for their betterment. However, the main issue is that enterprises cannot work in a non-regulated environment.
\item True decentralization: public blockchain provides true decentralization. This is something that is quite absent in private blockchain networks. As everyone has a copy of the ledger, it creates a distributed nature as well. Basically, in this type of blockchain, there is not a centralized entity. Thus, the responsibility of maintaining the network is solely on the nodes. With help from a consensus algorithm, they are updating the ledger, promoting fairness.
\item Full transparency: Public blockchain companies tend to design the platforms fully transparent to anyone on the ledger. It means that users can see the ledger anytime they want. So, there is no scope for any corruption or discrepancies. Anyhow, everyone has to maintain the ledger and participate in consensus.
\item Immutability: The public blockchain network is fully immutable. Once a block gets on the chain, there is no way to change it or delete it. So, it ensures that no one can alter a particular block to benefit from others.
\end{itemize}

\subsubsection{Consensus for Public Blockchain}\label{sec:consensoPublica}
Due to the uncertainties regarding the participants, public blockchains generally adopt mining-based consensus mechanisms. In these mechanisms, miners vie with each other for consensus leadership, using computational power, possession power over cryptocurrency or other election-relevant powers that cannot be monopolized such that the same knots always come out victorious) \cite{greve2018blockchain}.

Compensation to these miners for their work is often cryptocurrency. These incentives are critical to preventing Byzantine attacks by solving the fundamental challenge of agreement globally. Currently, proof of work is one of the few prosperous and resilient consensuses approaches to Sybil attacks \cite{douceur2002sybil} (impersonation attacks, when malicious users become impersonate others).

In the mining process, new transactions are vilified by the nodes in the whole system known as “miners” before being added to the blockchain. Miners add new blocks on the chain or new transactions on the block by a consensus algorithm, which must be confirmed by the majority of all the nodes in the system, like a voting operation, as the valid data. Blockchain-based systems rely on miners to aggregate transactions into blocks and append them to the blockchain. Once a sufficient number of nodes confirms the transaction, it becomes a valid and permanent part of the database. 

\subsection{The pros and cons of public blockchain}\label{sec:prosConsPub}

One of the most significant advantages of a public blockchain is that there is no need for trust. Everything is recorded, public, and cannot be changed. Everyone is encouraged to do the right thing for the betterment of the network. There is no need for intermediaries \cite{blockgeeks2018deeper}.

The other significant advantage is security. The more decentralized and active a public blockchain is, the more secure it becomes. As more people work on the network, it becomes harder for any attack to succeed. It is nearly impossible for malicious actors to band together and control the network \cite{selfkeyOrg}.

The final piece of what makes public blockchain successful is transparency. All data related to transactions are open to the public for verification. The transparency of public blockchain increases potential use cases, such as decentralized identity \cite{Comstor2018}.

One of the biggest problems with public blockchain is speed. Public blockchains like Bitcoin are extremely slow, only managing to process seven transactions per second. Compare that to Visa, which can do 24,000 transactions per second, and we see where the problem is. Public blockchains are slow because it takes time for the network to reach a consensus. Additionally, the time needed to process a single block takes a long time compared to a private blockchain \cite{blockgeeks2018deeper}.

Public blockchains also face scalability concerns. With the current state of things, public blockchains cannot compete with traditional systems. The more a public blockchain is used, the slower it gets because more transactions clog the network. However, steps are being taken to remedy this problem. An example is Bitcoin’s Lightning Network \cite{selfkeyOrg}.

Lastly, energy consumption has been a concern when it comes to the public blockchain. Bitcoin’s algorithm relies on Proof-of-Work, which relies on using much electricity to function. That being said, other algorithms such as Proof-of-Stake use far less electricity \cite{selfkeyOrg}. 


\subsection{Permissioned Blockchain}\label{sec:blockchainPrivada}
Permissioned, federated, or private blockchains, on the other hand, perform a blockchain between a set of known and identified participants. A permissioned blockchain provides a way to protect the interactions between entities with a common goal but does not trust each other, like companies that trade funds, assets, or information. Relying on peer identities, one permissioned blockchain may use the traditional consensus of \acf{BFT} \cite{androulaki2018hyperledger}.

Federated blockchain has its known composition. It is formed by $n$ processes whose inputs and outputs are subject to permissions. Participants are identified, authenticated, and authorized. This blockchain model aims to serve better corporate or private interests where participants have well-defined roles and can even organize themselves into groups. Examples of permissioned blockchain are Hyperledger Fabric \cite{cachin2016architecture} and some other projects \cite{cachin2017blockchain}.

As enterprises need privacy, permissioned blockchain use cases seem a perfect fit in this case. Without proper privacy, their competition can enter the platforms and leaks valuable information to the press. This, in the long term, can influence the brand value greatly. So, in some instances, companies need privacy greatly \cite{101blockchains}.

The main characteristics of a public blockchain are:

\begin{itemize}
\item High efficiency: Compared with a public blockchain, which tends to lack inefficiency because it introduces everyone to the network. As a result, when more people try to use the features, it takes up many resources that the platforms cannot back up. Thus, it slows down rapidly. On the other hand, private blockchain only allows a handful of people in the network. In many cases, they even have specific tasks to complete. So, there is no way they can take up extra resources and slow down the platform.
\item Full privacy: Private blockchain tends to focus on privacy concerns. Enterprises always deal with security ad privacy issues. More so, they also deal with such sensitive information daily. If even one of them gets leaked, it can mean a massive loss for the company.
\item Empowering enterprises: Private blockchain solutions work to empower the enterprises as a whole rather than individual employees. In reality, companies do need great technology to back up their processes. More so, these solutions are mainly for the internal systems of an enterprise. This is one of the best use cases of the private blockchain.
\item Stability: Private blockchain solutions are stable. Basically, in every blockchain platform, users have to pay a certain fee to complete a transaction. However, the fee can increase significantly in public platforms due to the pressure of nodes requesting transactions. When there are too many transaction requests, it takes time to complete them. More so, as time increases, the fee increases drastically.
\item Low fees: In private blockchain platforms, the transaction fees are meager. Unlike public blockchain platforms, the transaction fee does not increase based on the number of requests. So, no matter how many people request a transaction, the fees will always stay low and accurate.
\item Saves money: In reality, private blockchain saves much money. Maintaining a private blockchain is relatively simple compared to public blockchains. Private blockchain platforms take up only a few resources.
\item No illegal activity: Private blockchain platforms have authentication processes before any user logs into the network. What this process does is filter any intruders trying to get into the system.
\item Regulations: For enterprise companies that have to follow many rules and regulations, private blockchain outlines all the rules, and the peer nodes have to follow them.
\end{itemize}

\subsubsection{Consensus for permissioned blockchain}\label{sec:consensoPrivada}
Since it is a controlled network with $n$ known participants and identified by the federation, classic \acf{BFT} protocols and deterministic Byzantine consensus can be adapted to the blockchain \cite{androulaki2018hyperledger}.

In addition, there is no need to use incentives to the agreement, as the federation of stakeholders can establish its financial model of remuneration. Incentives, however, may be used for other purposes but, different from the evidence-based consensus, they are not essential to consensus \cite{greve2018blockchain}.

In the \ac{BFT} literature, replication consistency is maintained by two principles:

\begin{itemize}
\item No mistake: Leaders are prevented from making mistakes, so only one possible proposal per leader per rating.
\item Proposal Security: A (higher-ranked) proposal can extend, but not modify, any lower-ranking compromised log prefix \cite{abraham2017blockchain}.
\end{itemize}

\subsection{The pros and cons of permissioned blockchain}\label{sec:prosConsPriv}

A significant advantage of permissioned blockchain is speed. Permissioned blockchains have far fewer participants, meaning it takes less time for the network to reach a consensus. As a result, more transactions can take place. Permissioned blockchains can process thousands of transactions per second. Comparing that to Bitcoin's seven transactions per second is a massive difference \cite{di17blockchain}.

Permissioned blockchains are also far more scalable. Since only a few nodes are authorized and responsible for managing data, the network can process more transactions. The decision-making process is much faster because it is centralized \cite{selfkeyOrg}.

However, the centralization of permissioned blockchain is one of its most significant disadvantages. Blockchain was built to avoid centralization and permissioned blockchain inherently becomes centralized due to its private network \cite{blockgeeks2018deeper}.

Trust is also a more significant issue when it comes to permissioned blockchain. The credibility of a permissioned blockchain network relies on the credibility of the authorized nodes. They need to be trustworthy as they are verifying and validating transactions. The validity of records also can’t be independently verified \cite{blockgeeks2018deeper}.

Security is another concern with permissioned blockchain. With fewer nodes, it is easier for malicious actors to gain control of the network. Unfortunately, a permissioned blockchain is far more at risk of being hacked or having data manipulated \cite{abraham2017blockchain}.

\begin{table}[H]
\caption{Public Vs. Permissioned blockchain \cite{101blockchains}:}
\label{table:pubVsPriv}
    \begin{tabular}{|l|p{5.61cm}|p{5.61cm}|}
        \hline 
        \thead{} & \thead{Public blockchain } & \thead{Permissioned blockchain}\\
        \hline 
        Access & Anyone & Single Organization\\
        \hline
        Authority & Decentralized & Partially decentralized\\
        \hline
        Transaction Speed & Slow & Fast\\
        \hline
        Consensus & Permissionless & Permissioned\\
        \hline
        Transaction Cost & High & Low\\
        \hline
        Data Handling & Read and Write access for anyone & Read and Write access for a single organization\\
        \hline
        Immutability & Full & Partial\\
        \hline
        Efficiency & Low & High\\
        \hline
    \end{tabular}
\end{table}
