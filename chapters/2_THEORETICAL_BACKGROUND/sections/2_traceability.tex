\section{Traceability}\label{sec:traceability}

\citeonline{gryna1998juran} defines traceability as the ability to preserve the identity of the products and their origins so that the collection, documentation, and maintenance of information related to all processes in the production chain must be ensured. For a food product, traceability represents the ability to identify where and how it was grown, as well as the ability to track its post-harvest history and to identify the processes performed at each step in the production chain through records. Traceability is required primarily for \cite{horiuchirastreabilidade}:
\begin{itemize}
\item Improve credibility with customers and consumers.
\item Ensure that only quality materials and components are present in the final product.
\item Better allocate responsibilities.
\item Identify products that are distinct but may be confused.
\item Enable the return of defective or suspect products.
\item Find the causes of failures and take steps to repair them at the lowest possible cost.
\end{itemize}

Consumers consider traceability as part of a standard protection package when purchasing products. Traceability can improve credibility in this scope since all the related providers and dealers or another agent between the raw manufacturers and the final consumer can be tracked. In a traceability system, the responsibility papers are well defined. A traceability system allows users to track products by providing information about them (e.g., originality, components, or locations) during production and distribution. Suppliers and retailers typically require independent, government-certified traceability service providers to inspect products throughout the supply chain. Vendors and retailers request traceability services for different purposes. Suppliers want to receive certificates to showcase their products. Retailers want to verify the origin and quality of products \cite{lu2017adaptable}.


In \cite{opara2003traceability} six crucial elements are to be considered for traceability:

\begin{itemize}
\item Product Traceability: Determines the location of a product at any stage of the production chain to facilitate logistics, inventory management, product recall, and information disclosure to consumers and customers.
\item Process Traceability: Identifies the type and sequence of activities that affected a particular product. This includes any interactions between the product and physical / mechanical, chemical, and environmental factors that result in the transformation of raw material into value-added products.
\item Genetic traceability: determines the genetic constitution of the product.
\item Input traceability: Determines the type and source of input, such as fertilizers and livestock.
\item Traceability of Diseases and Pests: Tracks the epidemiology of pests such as bacteria and viruses.
\item Measurement traceability: determines the measurement instruments and specifying the environmental, geospatial, and temporal factors that influence data quality.
\end{itemize}

Supply chain visibility, or traceability,  is one of the key challenges encountered in the business world. Most companies have little or no information about their own second and third-tier suppliers. Transparency and end-to-end visibility of the supply chain can help shape product, raw material, test control, and end product flow, enabling better operations and risk analysis to ensure better chain productivity \cite{abeyratne2016blockchain}.

Folinas et al. \cite{folinas2006traceability} identified that the efficiency of a traceability system depends on the ability to track and trace each asset and logistics unit in a way that enables continuous monitoring from firstly processed until final clearance by the consumer.

\citeonline{aung2014traceability} and \citeonline{golan2004traceability} set three main traceability objectives, namely: (1) better supply chain management, (2) product differentiation and quality assurance, and (3) better identification of non-compliant products. An additional objective is to maintain assurance of traceability following applicable regulations and standards. A complete traceability system will include components that manage \cite{vargas2017trazabilidad}:

\begin{enumerate}
\item Identification, marking and assignment of traceable objects, parties and locations.
\item Automatic capture (by scanning or reading) of movements or events involving an object.
\item Record and share traceability data, internally or with parts of a supply chain, to gain visibility of what has occurred.
\end{enumerate}

Traceability systems typically store information in standard databases controlled by service providers. This centralized data storage becomes a single point of failure and risks tampering. Consequently, these systems result in trust problems, such as fraud, corruption, tampering, and falsifying information. Likewise, a centralized system is vulnerable to collapse \cite{tian2017supply}.

Nowadays, a new technology called the blockchain presents a whole new approach based on decentralization. Blockchain enables end-to-end traceability, bringing a standard technology language to the supply chain while allowing consumers to access the asset's history of these products through a software application. This characteristic is provided by an immutable register ledger that facilitates recording transactions and tracking assets across a network \cite{galvez2018future}.