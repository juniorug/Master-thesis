\section{Hyperledger}\label{sec:hyperledger}

As the popularity of public blockchain and a few other derivative technologies grew, interest in applying the underlying technology of the blockchain, distributed ledger and distributed application platform to more innovative enterprise use cases also grew. However, many enterprise use cases require performance characteristics that the permissionless blockchain technologies are unable (presently) to deliver. In addition, in many use cases, the identity of the participants is a hard requirement, such as in the case of financial transactions where Know-Your-Customer (KYC) and Anti-Money Laundering (AML) regulations must be followed \cite{POLGE2020}.

For enterprise use, is necessary to consider the following requirements:

\begin{itemize}
\item Participants must be identified/identifiable;
\item Networks need to be permissioned;
\item High transaction throughput performance;
\item Low latency of transaction confirmation;
\item Privacy and confidentiality of transactions and data pertaining to business transactions.
\end{itemize}

These requirements have a good fit with the required non-functional requirements needed for a SCM project and the ones specified in table~\ref{table:rnf}. While many early blockchain platforms are currently being adapted for enterprise use, Hyperledger Fabric has been designed for enterprise use from the outset. 

The Hyperledger Project is a collaborative effort to create an enterprise-grade, open-source distributed ledger framework and code base. It aims to advance blockchain technology by identifying and realizing a cross-industry open standard platform for distributed ledgers, which can transform the way business transactions are conducted globally. Hyperledger was established as a project of the Linux Foundation in early 2016 \cite{cachin2016architecture}.

Hyperledger Fabric is an implementation of a distributed ledger platform for running smart contracts, leveraging familiar and proven technologies, with a modular architecture allowing pluggable implementations of various functions. It is one of multiple projects currently in incubation under the Hyperledger Project \cite{cachin2016architecture}. Hyperledger Fabric is a widely used permissioned blockchain primarily in enterprise setting to make transactions between multiple business more efficient. This implementation defines an asset or assets and the transaction instructions for modfying them. Members of each permissioned network interact with the ledger using chaincode. The Membership identity service manages IDs and authenticate participants. Also, the Access Control List provides additional layers of permission \cite{blockgeeks2018deeper}.

There are two types of nodes in Hyperledger Fabric: peer nodes and ordering nodes. Peer nodes are responsible for executing and verifying transactions, while ordering nodes are responsible for ordering transactions and propagating the correct history of events to the network. This is done to increase efficiency and scalability by allowing peer nodes to batch and process multiple transactions simoutaneously \cite{buterin2016smart}. 

Fabric Ledger is comprised of two components: A blockchain log to stores the immutable sequenced record of transactions in blocks and a State Database to mantain the blockchain's current state. In a public blockchain there is no state database. It means that the current state of the chain is always calculated by going through all the transactions in the ledger. For speed and efficiency sake the Fabric stores the current state  and allows members of the network to query it as a SQL-like transactions \cite{blockgeeks2016blockchain}. 

The logs purpose is to trap an asset providence or place of origin as it exchange among multiple parties. To track an assets' provenance means to track where and when it was created as well as every time it was exchanged. Tracking an asset's providence is extremely important in the world of business because it ensures that the business selling an item possesses a chain of titles verifying their ownership of it. In typical databases, where only the current state of an asset is kept and not a log of all transactions, tracking an asset providence becomes very difficult. Add to this the fact that transacting businesses each keep an incomplete record of the asset transaction and it becomes nearly impossible \cite{blockgeeks2016blockchain}.

Hyperledger fabric uses Private Channnels to solve the problem of sensitive information which could be seen for other parties or competitors in a public blockchain. Private Channels are restricted message paths that can be used to provide transaction privacy and confidentiality for specific subset of network members. All data including transactions, member and channel information on the channel are invisible and inacessible to any network members not explicitly granted access to that channel. This allows competing business interests and any groups that require private confidential transactions to coexist on the same permissions network \cite{brabbani2017hashing}.