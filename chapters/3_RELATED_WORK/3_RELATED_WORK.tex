\xchapter{Related Work}{This section presents some conventional and Blockchain-based SCM systems, their main characteristics, and how this work presents a different approach.}\label{chap:RelatedWork}

% É recomendável utilizar `\acresetall' no início de cada capítulo para reiníciar o contator de referências às siglas.
\acresetall 

The rapid growth of Internet technologies allowed the onset of lots of technologies applied in traceability systems. In order to solve some problems with Supply Chain traceability, many \ac{IoT} technologies, such as \ac{RFID} and wireless sensor network-based architectures, have been applied. However, these technologies do not guarantee that the information shared by supply chain members in the traceability systems can be trusted \cite{tian2017supply}.

Blockchain presents a whole new approach based on decentralization. Nonetheless, being in its early stages has some challenges to deal with, in which scalability and performance become mainly defiance to face the massive amount of data in the real world. Through this technology, some solutions have been raised, as follows.

\section{Traditional Systems} \label{sec:TraditionalSystems}

Microsoft's Dynamics 365 is excellent for simple SCM needs, and it is just as accessible as every other Microsoft suite on the market. Dynamics integrates with third-party management systems, but it is primarily for "simple needs." Microsoft's offering is not so great for complex supply chain demands - but then, the  more significant majority of organizations have not got overly complex networks \cite{bellu2018microsoft}.

Plex Systems was one of the very first supply chain and manufacturing \ac{SaaS} ERP systems. The software is a cloud-based SCM that is very popular with industry-leading companies - especially in the aerospace and automotive industries. Unfortunately, though, for all its maturity and complex capabilities, Plex lets organizations down with its inability to support several implementation partners \cite{plex}.

Oracle NetSuite is a cloud-based supply chain and ERP system for the less complex needs of moderately-sized companies and most SCM and ERP systems \cite{rolling2016using}. As ERP focused system, this project is focused on business management instead of SCM traceability and information transparency \cite{rolling2016using}.

SAP Supply Chain Management harnesses AI and the Internet of Things to provide visibility and analytics to help the users plan, source, and deliver the goods and materials.  This is a real-time supply chain planning software that connects stakeholders \cite{snapp2010discover}.

These systems typically are ERP solutions focused on business management, controlled by service providers.  This centralized data  storage becomes a single point of failure and risks tampering. As a centralized organization, it can become a vulnerable target for  bribery, and then the whole system can not be trusted anymore. Also, as proprietary systems, there are some concerns about reliability, security, decentralization, immutability, transparency, and lack of trust among participants. There is no trusted third party to ensure data reliability. 

\section{Blockchain-based Systems} \label{sec:BlockchainBasedSystems}

There are advantages of applying the Blockchain concept to a supply chain. One of the most important is that all stakeholders involved in the supply chain are motivated to demonstrate to customers the superior quality of their methods and products \cite{lu2017adaptable}. In addition, a Blockchain can be used as a marketing tool. As Blockchains are fully transparent and participants can control the assets in them, they can be used to enhance the image and reputation of a company \cite{van2007essentials}, drive loyalty among existing customers \cite{pizzuti2015global}, and attract new ones \cite{svensson2009transparency}. In fact, companies can easily distinguish themselves from competitors by emphasizing transparency and monitoring product flow along the chain. 

In \cite{tian2017supply}, it is proposed a system that combines HACCP (a food safety protocol), Blockchain, and IoT in order to provide food safety traceability. Each member can add, update and check the information about the product on the Blockchain as long as they register as a user in the system. Each product also has a unique digital cryptographic identifier that connects the physical items to their virtual identity in the system. This virtual identity can be seen as a product information profile.

The Everledger Diamonds project provides a Blockchain-based solution to facilitate tracking from mine to consumer, enabling easier compliance against increasingly strict measures from diamonds produced \cite{crosby2016blockchain}.

IBM Food Trust is a pilot project motivated by food contamination scandals worldwide. The main goal is to tackling food safety in the supply chain using Blockchain technology. This platform tracked pork in China and mangoes in the Americas \cite{kamath2018food}.

\section{Comparison with the presented work} \label{sec:Comparison}
These projects are focused on specific products only and are closed projects. Still, there is a general lack of standards for implementing a Blockchain approach for traceability. A Blockchain must be universal and adaptable to specific situations \cite{valenta2017comparison}. In addition, the need to agree on a particular type of Blockchain to be used puts the parties under pressure. 

Compared to traditional systems, the great difference of this work is to provide the non-functional requirements acquired by using the blockchain:

\begin{itemize}
\item More reliable operations;
\item More democratic transactions;
\item Process optimization;
\item Ease of coordination between companies;
\item Recording data in chronological order;
\item Cost reduction;
\item Distributed and autonomous platform.
\end{itemize}

In relation to blockchain-based systems:

\begin{itemize}
\item Not specific to a product type;
\item Open source project and not closed;
\end{itemize}

Our work is intended to provide a Blockchain-based platform to facilitate the development of applications for traceability in supply chain management.

\begin{table}[H]
\caption{Related Works and their main characteristics, strategies and results.}
\label{table:userStories}
    \begin{tabular}{|l|p{2.1cm}|p{3.2cm}|p{4.5cm}|}
    \hline 
    \thead{Related work} & \thead{Blockchain} & \thead{Solution} & \thead{Main Deficiency} \\ 
    \hline 
    Microsoft Dynamics 365  & No & Simple SCM needs & Centralized solution. Not a distributed and autonomous platform.\\
    \hline
    Plex System  & No & SaaS ERP system & Centralized solution. Not easy to coordinate between companies. High cost. Private System. \\
    \hline 
    Oracle  NetSuite  & No & cloud-based supply chain and ERP system & Centralized solution. High cost. Private System.\\
    \hline 
    SAP SCM  & No & AI and the Internet of Things & High cost. Private System.\\
    \hline 
    Tian  & Yes & Oroposed system that combines HACCP, Blockchain, and IoT & Theoretical application. Food related only.\\
    \hline 
    Everledger Diamonds  & Yes & Blockchain-based solution & Product specific. Not an open source project. High cost. \\
    \hline 
    IBM Food Trust  & Yes & Blockchain-based solution & Product specific. Not an open source project. High cost.\\
    \hline
    SCM-BP  & Yes & Blockchain-based solution & Relies on a minimum amount of participation to obtain a reliable forecast.\\
    \hline
    \end{tabular}
\end{table}