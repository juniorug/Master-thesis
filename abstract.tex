%%%%%%%%%%%%%%%%%%%%%
% Resumo em Portugues
%%%%%%%%%%%%%%%%%%%%%

\resumo
Os avanços nas tecnologias da informação e comunicação reduziram as barreiras físicas, políticas e culturais entre as nações. Essas tecnologias permitiram a globalização ao acesso de matérias-primas, bens e serviços. A complexa rede de relacionamentos que envolve quem fornece materiais, fabricam componentes ou subprodutos, montam ou misturam as partes e entregam o produto final no mercado é conhecida como cadeia de suprimentos. O rápido crescimento das tecnologias da internet permitiu o surgimento de muitas soluções emergentes aplicadas em sistemas de rastreabilidade, na área da cadeia de suprimentos. No entanto, esses sistemas tendem a ser centralizados, monopolistas, assimétricos e opacos. Como consequência, essas aplicações estão susceptíveis à problemas de confiança como fraude, corrupção, adulteração e falsificação de informações. Da mesma forma, por ser um ponto único de falha, o sistema centralizado é vulnerável ao colapso. Atualmente, uma emergente tecnologia chamada blockchain apresenta uma nova abordagem baseada na descentralização. No entanto, por estar em seus estágios iniciais, ela tem alguns desafios a enfrentar, nos quais a ratreabilidate a escalabilidade e o desempenho se tornam principalmente um desafio para encarar a enorme quantidade de dados no mundo real. Este trabalho pretende fornecer uma estrutura baseada em blockchain para facilitar o desenvolvimento de aplicações para rastreabilidade no gerenciamento da cadeia de suprimentos, usando uma cadeia de recursos minerais como caso de uso.


% Palavras-chave do resumo em Portugues
\begin{keywords}
Blockchain; gerência de cadeia de suprimento; rastreabilidade; contratos inteligentes.
\end{keywords}

%%%%%%%%%%%%%%%%%%%
% Resumo em Ingles
%%%%%%%%%%%%%%%%%%%

\abstract
Advances in information and communication technologies have reduced the physical, political and cultural barriers between nations. These technologies have enabled globalization to access raw materials, goods and services. The complex web of relationships that provide materials manufacture the components, assemble or mix the parts and deliver the final product to market is known as the supply chain. The rapid growth of internet technologies allowed the onset of  lots of emerging technologies applied in traceability systems, in supply chain area. However, these systems tend to be centralized, monopolistic, asymmetric and opaque. As consequence, these systems results in trust problem, such as fraud, corruption, tampering and falsifying information. Likewise, by being a single point of failure, centralized system is vulnerable to collapse. Nowadays, a new technology called the blockchain presents a whole new approach based on decentralization. Nonetheless, by being in its early stages, it has some challenges to deal with, in which traceability, scalability and performance become  mainly  defiance to face the huge amount of data in the real world. This work is intended to provide a blockchain based framework in order to facilitate the development of applications for traceability in supply chain management using a mineral resource supply chain as use case.

% Palavras-chave do resumo em Ingles
\begin{keywords}
Blockchain; supply chain management; traceability; smart contracts.
\end{keywords}
