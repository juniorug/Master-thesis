\documentclass[en, msc, classic, a4paper]{ifbathesis}

%% Preambulo:
\usepackage[utf8]{inputenc}
\usepackage[english]{babel}
\usepackage{graphicx}
\usepackage{lipsum}
\usepackage{hyphenat}
% \usepackage[usenames, dvipsnames, table]{xcolor}
\usepackage{float}
\usepackage{booktabs}
\usepackage{pifont}
\usepackage{multirow}
\usepackage{listings} 
\usepackage{colortbl}
\usepackage{xfrac}
\usepackage{nameref}
\usepackage{cleveref}
\usepackage[FIGTOPCAP]{subfigure}
\usepackage[printonlyused, withpage]{acronym}
\usepackage{graphicx,url}
%\usepackage[hidelinks,breaklinks]{hyperref}
%\PassOptionsToPackage{hyphens}{url}
\usepackage{makecell}
\renewcommand\theadfont{\bfseries\sffamily}
\usepackage{minted}

% ADD SUBSUBSUBSECTION
\newcommand{\subsubsubsection}[1]{\paragraph{#1}\mbox{}\\}
\setcounter{secnumdepth}{4}
\setcounter{tocdepth}{4}

\newenvironment{dedication}
  {\clearpage           % we want a new page
   \thispagestyle{empty}% no header and footer
   \vspace*{\stretch{6}}% some space at the top 
   \itshape             % the text is in italics
   \raggedleft          % flush to the right margin
  }
  {\par % end the paragraph
   \vspace{\stretch{0}} % space at bottom is three times that at the top
   \clearpage           % finish off the page
  }
  
\university{Instituto Federal da Bahia}
\address{Salvador}
\institute{Departamento de Pós-Graduação e Qualificação}
\library{Biblioteca Professor Raul Varella Seixas}
\program{Programa de P\'{o}s-Gradua\c{c}\~{a}o em Engenharia de Sistemas e Produtos}
\majorfield{Engenharia de Sistemas e Produtos}
\title{A blockchain framework for traceability in supply chain management}

% Data da defesa
% e.g. \date{19 de fevereiro de 2013}
\date{15 de maio de 2021}
% e.g. \defenseyear{2013}
\defenseyear{2021}

% Autor
\author{Edivaldo Mascarenhas Ferreira de Jesus Júnior}

% Orientador(a)
\adviser{Manoel Carvalho Marques Neto}
\coadviser{Allan Edgard Silva Freitas}

%% Inicio do documento
\begin{document}
\ppgespfrontpage
\frontmatter
\ppgesppresentationpage

\begin{dedication}
\vspace*{\fill}
    This work is dedicated to the memory of \\ Edivaldo M. F. de Jesus (1951--2021)
    \vspace*{\fill}

\end{dedication}

\input{abstract}
\tableofcontents
\listoffigures
\listoftables
\chapter*{List of Acronyms}
\begin{acronym}[PPGESP]
    \acro{PPGESP}{Programa de Pós-Graduação em Engenharia de Sistemas e Produtos}
    \acro{CNPq}{Conselho Nacional de Desenvolvimento Científico e Tecnológico}
    \acro{BG}{Byzantine generals}
    \acro{PoW}{proof-of-work}
    \acro{BFT}{Byzantine fault tolerance}
    \acro{SC}{smart contract}
    \acro{SCM}{Supply Chain Management}
    \acro{TAM}{Technology Acceptance Model}
    \acro{SCM-BP}{Supply Chain Management - Blockchain Platform}
\end{acronym}

%% Parte textual
\mainmatter

%Seções
\xchapter{Introduction}{}

% É recomendável utilizar `\acresetall' no início de cada capítulo para reiníciar o contator de referências às siglas.
\acresetall 

Hundreds of years ago, supply chains were reasonably straightforward. Mines and farms provided natural resources to skilled craftsmen like blacksmiths and tailors, creating and selling finished products. Today's supply chains are much more complicated, fragmented, and difficult to understand. Hundreds or even thousands of supplies worldwide contribute to making and ship products purchased by a customer. Most of the time, various companies don't know each other, and a final consumer likely doesn't know anything about how, where, when, or under what condition the products passed through. This is not just a problem for consumers. Today's supply chains are so complex that even big industry players have difficulty tracking how their goods get made.

In order to solve problems that come with this complexity, such as supply chain visibility and traceability, many systems have been developed. However, these systems typically store information in standard databases controlled by service providers. This centralized data storage becomes a single point of failure and risks tampering. Products' origin information may be essential, and a central server may be changed, generating doubt about confiability in that spot. As a centralized organization, it can become a vulnerable target for bribery, and then the whole system can not be trusted anymore.

Blockchain and smart contracts could make supply chain management more straightforward and transparent. The idea is to create a single source of information about products and supply chains via a global ledger. Each component would have its own entry on the blockchain that gets tracked over time. Both untrust companies could then update the status of goods in real-time. The result is that once the clients receive their products, they could track every piece back to its manufacturer. Theoretically, users could trace the supply chain all the way back to the mines where the raw materials came from \cite{greve2018blockchain}.

Companies can also use the blockchain supply chain as a single source of truth for their products. They can manage and monitor risks within the supply chain, ensure the quality of delivery parts and track the delivery status. Additionally, companies can use smart contracts to manage and pay for the supply chain autonomously. This would reduce the need for large contract invoices on the back-and-forth of refund requests for faulty components. Those same smart contracts could assist with shipping and logistics tracking valuable products as they travel around the world. Using blockchain, companies can finally have a complete picture of their products at every stage in the supply chain, bringing transparency to the production process while reducing the cost of manufactured goods.

This work presents \acf{SCM-BP}, a generic framework intended to be used in any supply chain correlated to assets and products. It also presents a use case of this framework applied.

This paper is structured as follows: Chapter 2 presents several technologies involved in preparing this dissertation, introduces essential concepts of the Computer area in which the context of this project is inserted, and presents business concepts related to supply chain management. Chapter 3 shows correlated works. Chapter 4 presents the solution, its architecture, and details of the executed implementation. Chapter 5 exposes the validation and its results. The last chapter presents the conclusions and future work.
\xchapter{Theoretical Background}{In this section the main concepts studied are presented, which provided subsidies for the development of the proposed project.}\label{chap:Theoretical}


%\input{chapters/2_THEORETICAL_BACKGROUND/sections/0_generalContext.tex}
\input{chapters/2_THEORETICAL_BACKGROUND/sections/1_blockchain.tex}
\input{chapters/2_THEORETICAL_BACKGROUND/sections/2_fundamentalsOfBlockchain.tex}
\section{Public Blockchain Versus Permissioned Blockchain}\label{sec:versus}

Blockchain networks can be categorized into two groups: public (or permissionless) blockchain, and private (or federated or permissioned) blockchain (with permission and controlled access) \cite{greve2018blockchain}.

Since the beginning of blockchain technology, people have debated about public vs permissioned blockchain. In an enterprise environment, it’s actually really important to know the big differences between these two. Basically, public and permissioned blockchain examples play a huge role in the companies looking for the perfect blockchain type for their solutions \cite{101blockchains}. The table \ref{table:pubVsPriv} presents a comparison between public and permissioned blockchain.

\subsection{Public blockchain}\label{sec:blockchainPublica}
On a public or permissionless blockchain, any person can participate without a specific identity. Basically, there are no restrictions when it comes to participation. Public blockchains typically involve a native cryptocurrency and often use \acf{PoW} consensus and economic incentives \cite{androulaki2018hyperledger}. Public blockchain can be audited by anyone, and each node has as much transmission power as any other. For a transaction to be considered valid, it must be authorized by all nodes constituents via the consensus process. As long as each node meets protocol-specific stipulations, their transactions can be validated and thus added to the chain \cite{Comstor2018}.

In the public blockchain all the participants have equal rights no matter what. People can join in and participate in consensus, transact with their peers as they please. Everyone can see the ledger as well, thus maintaining transparency at all times \cite{101blockchains}.

As the P2P network node set is unknown, its membership is dynamic, allowing random node entrances and exits and also anonymity of them. Blockchain can act in global scale, without the control of its participants, who do not even trust each other mutually. Are examples of public blockchain the Bitcoin network, the Ethereum and several other cryptocurrencies \cite{bashir2018mastering, antonopoulos2017mastering}.

If a fully decentralized network system is required, then public blockchain is the way to go. However, it can get a bit problematic when trying to incorporate a public blockchain network with the enterprise blockchain process \cite{101blockchains}.

The main characteristics of a public blockchain are:

\begin{itemize}
\item High security;
\item Open environment;
\item Anonymous nature;
\item No regulations;
\item True decentralization;
\item Full transparency;
\item Immutability.
\end{itemize}

\subsubsection{Consensus for Public Blockchain}\label{sec:consensoPublica}
Due to the uncertainties regarding the participants, public blockchains generally adopt mining-based consensus mechanisms. In these mechanisms miners vie with each other for consensus leadership, using computational power, possession power over cryptocurrency or other election-relevant powers that cannot be monopolized such that the same knots always come out victorious) \cite{greve2018blockchain}.

Compensation to these miners for their work is often cryptocurrencies. These incentives are critical to prevent Byzantine attacks by solving the fundamental challenge of agreement on a global scale. Currently proof of work is one of the few successful and resilient consensus approaches to Sybil attacks \cite{douceur2002sybil} (impersonation attacks, when malicious users become impersonate others).

\subsection{Permissioned Blockchain}\label{sec:blockchainPrivada}
Permissioned, federated or private blockchains, on the other hand, perform a blockchain between a set of known and identified participants. A permissioned blockchain provides a way to protect the interactions between a group of entities that have a common goal but that don't totally trust each other, like companies that trade funds, assets or information. Relying on peer identities, one permissioned blockchain may use the traditional consensus of \acf{BFT} \cite{androulaki2018hyperledger}.

Federated blockchain has its known composition. It is formed by $n$ processes whose inputs and outputs are subject to permissions. Participants are identified, authenticated and authorized. This model of blockchain aims to better serve corporate or private interests where participants have well-defined roles and can even organize themselves into groups. Examples of permissioned blockchain are Hyperledger Fabric \cite{cachin2016architecture} and some other projects \cite{cachin2017blockchain}.

As enterprises need privacy, permissioned blockchain use cases seems a perfect fit in this case. Without proper privacy, their competition can enter the platforms and leaks valuable information to the press. This, in the long term, can influence the brand value greatly. So, in certain cases, companies need privacy greatly \cite{101blockchains}.

The main characteristics of a public blockchain are:

\begin{itemize}
\item High efficiency;
\item Full privacy;
\item Empowering enterprises;
\item Stability;
\item Low fees;
\item Saves money;
\item No illegal activity;
\item Regulations;
\end{itemize}

\subsubsection{Consensus for Permissioned Blockchain}\label{sec:consensoPrivada}
Due to the fact that it is a controlled network with $n$ known participants and identified by the federation, classic \acf{BFT} protocols and deterministic Byzantine consensus can be adapted to the blockchain \cite{androulaki2018hyperledger}.

In addition, there is no need to use incentives to agreement, as the federation of stakeholders can establish its own financial model of remuneration. Incentives, however, may be used for other purposes but, different from evidence-based consensus, they are not essential to consensus \cite{greve2018blockchain}.

In the \ac{BFT} literature, replication consistency is maintained by two principles:

\begin{itemize}
\item No mistake: Leaders are prevented from making mistakes, so there is only one possible proposal per leader per rating.
\item Proposal Security: A (higher-ranked) proposal can extend, but not modify, any lower-ranking compromised log prefix \cite{abraham2017blockchain}.
\end{itemize}


\begin{table}[H]
\caption{Public Vs Permissioned Blockchain: The Comparison Table \cite{101blockchains}}
\label{table:pubVsPriv}
    \begin{tabular}{|l|p{5.61cm}|p{5.61cm}|}
        \hline 
        \thead{} & \thead{Public blockchain } & \thead{Permissioned blockchain}\\
        \hline 
        Access & Anyone & Single Organization\\
        \hline
        Authority & Decentralized & Partially decentralized\\
        \hline
        Transaction Speed & Slow & Fast\\
        \hline
        Consensus & Permissionless & Permissioned\\
        \hline
        Transaction Cost & High & Low\\
        \hline
        Data Handling & Read and Write access for anyone & Read and Write access for a single organization\\
        \hline
        Immutability & Full & Partial\\
        \hline
        Efficiency & Low & High\\
        \hline
    \end{tabular}
\end{table}
\input{chapters/2_THEORETICAL_BACKGROUND/sections/4_smartContracts.tex}
\section{Supply Chain Management}\label{sec:scm}

The complex web of relationships that provide materials manufacture the components, assemble or mix the parts and deliver the final product to market is known as the supply chain \cite{buurman2002supply}.

Management is on the verge of a major breakthrough in understanding how industrial company success depends on the interactions between the flows of information, materials, money, manpower, and capital equipment. The way these five flow systems interlock to amplify one another and to cause change and fluctuation will form the basis for anticipating the effects of decisions, policies, organizational forms, and investment choices \cite{forrester1958industrial}.

The term \ac{SCM} has risen to prominence over the past fifteen years \cite{cooper1997supply}. For example, at the 1995 Annual Conference of the Council of Logistics Management, 13.5\% of the concurrent session titles contained the words "supply chain." At the 1997 conference, just two years later, the number of sessions containing the term rose to 22.4\%. Moreover, the term is frequently used to describe executive responsibilities in corporations \cite{la1997supply}. SCM has become such a "hot topic" that it is difficult to pick up a periodical on manufacturing, distribution, marketing, customer management, or transportation without seeing an article about SCM or SCM-related topics \cite{ross1997competing}.

The definition of supply chain seems to be more common across authors than the definition of "supply chain management" \cite{cooper1993characteristics, la1994emerging, lambert1998fundamentals}. La Londe and Masters proposed that a supply chain is a set of firms that pass materials forward. Normally, several independent firms are involved in manufacturing a product and placing it in the hands of the end user in a supply chain—raw material and component producers, product assemblers, wholesalers, retailer merchants and transportation companies are all members of a supply chain \cite{la1994emerging}. By the same token, Lambert, Stock, and Ellram define a supply chain as the alignment of firms that brings products or services to market. 

Another definition notes a supply chain is the network of organizations that are involved, through upstream and downstream linkages, in the different processes and activities that produce value in the form of products and services delivered to the ultimate consumer \cite{christopher2017logistics}. In other words, a supply chain consists of multiple firms, both upstream (i.e., supply) and downstream
(i.e., distribution), and the ultimate consumer. 

As a summarization, \cite{mentzer2001defining} defines a supply chain as a set of three or more entities (organizations or individuals) directly involved in the upstream and downstream flows of products, services, finances, and/or information from a source to a customer. 


To begin with, the starting point of a supply chain is the extraction of raw materials and how they are first processed (preprocessed) by suppliers for delivery in the next step. The next step is called manufacturing, where the conversion process for raw materials takes place. Following this, the constructed products are passed to the distributors who are responsible for allocating them to multiple different intermediaries, such as wholesalers and retailers. Distributors also maintain an active inventory of products, as previous products are connected to suppliers. Subsequently, wholesalers do not sell products directly to the public, but to other retailers, whereas retailers dispose products purchased to end users. Finally, Consumers are who buy or receive goods or services for personal needs or use and not for commercial resale or trade purposes \cite{litke2019blockchains}.

The manufacturer needs to validate crucial information about the natural resources they collected by reading and verifying all tags that the latter includes in its transactions and then proceeding to the proper execution of manufacturing step. New transactions with new information, such as manufacturer name, field experience and more, are sent after the internship has completed. Then the products are delivered to distributors. Distributors are able to sell products to wholesalers and retailers. This process is represented by transactions that contains important data, such as merchant and customer address, exchange value, product raw material quality, and more \cite{sauer2018extending}. 

As the distributors sell products to intermediates (generally not end users), they must check the information about the progress route until that stage, for example the raw material origin, manufactures company popularity, distributor address and others. Retailers can audit product's natural resource quality, and get the appropriate feedback before selling it to the consumer. After that, when a distributor send the product to the wholesaler, some details, such as manufacturer name, field experience and others, are submitted after the completion of acts in a similar way. Wholesaler needs to check these information and execute their selling to another wholesaler or retailer company. The same applies to the retailer companies. Finally, end users obtain the final product and should able to track and verify all aspects from the beginning of its supply chain journey \cite{litke2019blockchains}. 
\input{chapters/2_THEORETICAL_BACKGROUND/sections/6_traceability.tex}
\section{Supply Chain Management and blockchain}\label{sec:scmBlock}

In order to solve some problems with Supply chain visibility and traceability, many internet of things technologies, such as RFID and wireless sensor network-based architectures and hardware, has been applied. However, these technologies do not guarantee that the information shared by supply chain members in the traceability systems can be trusted. As a centralized organization, it can become a vulnerable target for bribery, and then the whole system can not be trusted anymore \cite{tian2017supply}.

Suppose companies A, B, and C exercise such roles in the chain: producer, distributor, and retailer. There is data centralization, single point of failure, and trust among the parties in traditional systems. Sometimes, a third-part as a certificate authority must guarantee some regulation, audition, or provide security among the participants. The use of blockchain technology can help solve these problems since blockchain provides better transparency, enhanced security, immutability of data, traceability, and decentralization.

Blockchain and distributed ledger technology underpinning cryptocurrencies such as Bitcoin represent a new and innovative technological approach to realizing decentralized, trustless systems. Indeed, the inherent properties of this digital technology provide fault-tolerance, immutability, transparency, and full traceability of the stored transaction records, as well as coherent digital representations of physical assets and autonomous transaction executions \cite{caro2018blockchain}.

Blockchain enables end-to-end traceability, bringing a standard technology language to the supply chain while allowing consumers to access the asset's history of these products through a web app. The need to track products across the complex supply chain from mineral prospecting and exploration to the end consumer is increasingly common: checking environmental impacts or simply ensuring transparency for consumers \cite{galvez2018future}.

Instead of storing data in a shadowy network system, blockchain allows all the goods' information to be stored in a shared and transparent system for all the members along the supply chain \cite{tian2017supply}. Monfared \cite{abeyratne2016blockchain} argued about the potential benefit of blockchain technology in the manufacturing supply chain. They proposed that the inherited characteristics of the blockchain enhance trust through transparency and traceability within any transaction of data, goods, and financial resources. Moreover, it could offer an innovative platform for a new decentralized and transparent transaction mechanism in industries and businesses.

Many members are among the supply chain, including suppliers, producers, manufacturers, distributors, retailers, consumers, and certifiers. Each of these members can add, update and check the information about the product on the blockchain as long as they register as a user in the system. Each product also has a unique digital cryptographic identifier that connects the physical items to their virtual identity in the system. Users in the system also have their digital profile, which contains information about their introduction, location, certifications, and association with products \cite{tian2017supply}.

Supply chain members can register themselves in the system through the register, providing credentials and a unique identity to the members. After registration, public and private cryptographic key pairs will be generated for each user. The public key can be used to identify the user's identity within the system, and the private key can be used to authenticate the user when interacting with the system. This enables each product can be digitally addressed by the users when being updated, added, or exchanged to the following user in the downstream position of the supply chain \cite{caro2018blockchain}.

All members of the business network agree with the information acquired in each transaction. Once consensus is reached, no permanent record can be changed. Each information provides critical data that can potentially reveal security issues with the product in question \cite{galvez2018future}.

A smart contract encodes the combination of services and other conditions defined in the contract. Therefore, the smart contract can automatically verify and apply these conditions. It also verifies all information required by regulation to enable automated verification of regulatory compliance \cite{lu2017adaptable}.  A smart contract, by default, has no owner. Once deployed, its author has no special privileges. Unauthorized users may accidentally trigger a function without permission. Therefore, smart contracts must have internal permission to verify contract permissions.

The smart contract structural design has a significant cost impact if the blockchain is public. The cost of contract implementation depends on its size because the code is stored in the blockchain, which implies data storage fees proportional to the size of the contract. Therefore, a structural design with more lines of code costs more. A blockchain consortium does not have a coin or token, so the monetary cost is not a problem. However, blockchain size is still a design concern because it grows with each transaction, and each participant has a replica of the entire blockchain. In addition, a more structural design may affect performance as it may require more transactions \cite{lu2017adaptable}.

A blockchain must be universal and adaptable to specific situations \cite{valenta2017comparison}. In addition, the need to agree on a particular type of blockchain to be used puts the parties under pressure. This is a significant disadvantage as blockchain technology is progressing rapidly, and predicting the best choice for the future is quite tricky \cite{galvez2018future}.

On the other hand, there are advantages to applying the blockchain concept to the enterprise supply chain. One of the most important is: all stakeholders involved in the supply chain (Raw material / Producer, Manufacturer, Distributor, Wholesaler, Retailer) are motivated by the need to demonstrate to customers the superior quality of their methods and products \cite{lu2017adaptable}. 

In addition to serving the functions of a traceability system, a blockchain can be used as a marketing tool.Because blockchains are fully transparent \cite{iansiti2017truth} and participants can control the assets in them \cite{liao2011food}, they can be used to enhance the image and reputation of a company \cite{van2007essentials}, drive loyalty among existing customers \cite{pizzuti2015global} and attract new ones \cite{svensson2009transparency}. 

Companies can easily distinguish themselves from competitors by emphasizing transparency and monitoring product flow along the chain. In addition, quickly identifying a source of contamination or loss can help protect a company's brand image \cite{mejia2010traceability} and alleviate the adverse impact of media criticism \cite{dabbene2011food}.

\subsection{Transparency}\label{sec:transparency}

The main goals of a blockchain are to facilitate information exchange, create a digital twin of information and its workflow, and validate the quality of assets as they move along the chain. These goals are achieved by allowing each participant to share claims, evidence, and assessments of each other's claims about the product. The journey of mineral resources along the supply chain is captured in a blockchain object called a "mineral bundle". At the journey's end, the package combines all information provided by stakeholders over the life of the mineral item. This information can be used to establish the provenance, quality, sustainability, and many other attributes of mineral assets \cite{martin2017technology}.

\subsection{Efficiency}\label{sec:efficiency}
Blockchain is an infrastructure that allows new transactions between players who do not yet know or trust each other. Smart contracts are instructions that interface with the blockchain protocol to automatically evaluate and possibly post transactions on the blockchain \cite{raskin2017law}. 

Similarly, smart libraries are specialized sets of blockchain-compatible functionality that can be used locally or privately or shared and licensed to other blockchain participants and agents. All participants meet at the blockchain, evaluate the statements made and notify their account holders when matches are found in quality, time, quantity, etc. Buyers and sellers are matched by a shared but reliable need for data combined and used by either party. So traceability does not have to wait for large company consortia to use patterns and semi-mandatory or concentrated business practices to access the information \cite{galvez2018future}.

\subsection{Safety and protection}\label{sec:Safety}
Blockchains can also be used to emit and manage the creation of unique cryptographic tokens \cite{nystrom1999pkcs}. Tokens can be made to represent the collateral value between two participants (for example, future production to be sent in a specific field lot). Tokens do not have to take value exchange for the financial settlement of invoices and contracts. Instead, they represent a license to publish information that becomes uniquely valued in proportion to the needs of others on the blockchain. The strategy around issuing these encryption tokens, which need not be implemented in the initial system, is still being defined \cite{galvez2018future}.
%\input{chapters/2_THEORETICAL_BACKGROUND/sections/6_supplyChainBlockchain.tex}
\xchapter{Related Work}{This section presents some conventional and Blockchain-based SCM systems, their main characteristics, and how this work presents a different approach.}\label{chap:RelatedWork}

% É recomendável utilizar `\acresetall' no início de cada capítulo para reiníciar o contator de referências às siglas.
\acresetall 

The rapid growth of Internet technologies allowed the onset of lots of technologies applied in traceability systems. In order to solve some problems with Supply Chain traceability, many \ac{IoT} technologies, such as \ac{RFID} and wireless sensor network-based architectures, have been applied. However, these technologies do not guarantee that the information shared by supply chain members in the traceability systems can be trusted \cite{tian2017supply}.

Blockchain presents a whole new approach based on decentralization. Nonetheless, being in its early stages has some challenges to deal with, in which scalability and performance become mainly defiance to face the massive amount of data in the real world. Through this technology, some solutions have been raised, as follows.

\section{Traditional Systems} \label{sec:TraditionalSystems}

Microsoft's Dynamics 365 is excellent for simple SCM needs, and it is just as accessible as every other Microsoft suite on the market. Dynamics integrates with third-party management systems, but it is primarily for "simple needs." Microsoft's offering is not so great for complex supply chain demands - but then, the  more significant majority of organizations have not got overly complex networks \cite{bellu2018microsoft}.

Plex Systems was one of the very first supply chain and manufacturing \ac{SaaS} ERP systems. The software is a cloud-based SCM that is very popular with industry-leading companies - especially in the aerospace and automotive industries. Unfortunately, though, for all its maturity and complex capabilities, Plex lets organizations down with its inability to support several implementation partners \cite{plex}.

Oracle NetSuite is a cloud-based supply chain and ERP system for the less complex needs of moderately-sized companies and most SCM and ERP systems \cite{rolling2016using}. As ERP focused system, this project is focused on business management instead of SCM traceability and information transparency \cite{rolling2016using}.

SAP Supply Chain Management harnesses AI and the Internet of Things to provide visibility and analytics to help the users plan, source, and deliver the goods and materials.  This is a real-time supply chain planning software that connects stakeholders \cite{snapp2010discover}.

These systems typically are ERP solutions focused on business management, controlled by service providers.  This centralized data  storage becomes a single point of failure and risks tampering. As a centralized organization, it can become a vulnerable target for  bribery, and then the whole system can not be trusted anymore. Also, as proprietary systems, there are some concerns about reliability, security, decentralization, immutability, transparency, and lack of trust among participants. There is no trusted third party to ensure data reliability. 

\section{Blockchain-based Systems} \label{sec:BlockchainBasedSystems}

There are advantages of applying the Blockchain concept to a supply chain. One of the most important is that all stakeholders involved in the supply chain are motivated to demonstrate to customers the superior quality of their methods and products \cite{lu2017adaptable}. In addition, a Blockchain can be used as a marketing tool. As Blockchains are fully transparent and participants can control the assets in them, they can be used to enhance the image and reputation of a company \cite{van2007essentials}, drive loyalty among existing customers \cite{pizzuti2015global}, and attract new ones \cite{svensson2009transparency}. In fact, companies can easily distinguish themselves from competitors by emphasizing transparency and monitoring product flow along the chain. 

In \cite{tian2017supply}, it is proposed a system that combines HACCP (a food safety protocol), Blockchain, and IoT in order to provide food safety traceability. Each member can add, update and check the information about the product on the Blockchain as long as they register as a user in the system. Each product also has a unique digital cryptographic identifier that connects the physical items to their virtual identity in the system. This virtual identity can be seen as a product information profile.

The Everledger Diamonds project provides a Blockchain-based solution to facilitate tracking from mine to consumer, enabling easier compliance against increasingly strict measures from diamonds produced \cite{crosby2016blockchain}.

IBM Food Trust is a pilot project motivated by food contamination scandals worldwide. The main goal is to tackling food safety in the supply chain using Blockchain technology. This platform tracked pork in China and mangoes in the Americas \cite{kamath2018food}.

\section{Comparison with the presented work} \label{sec:Comparison}
These projects are focused on specific products only and are closed projects. Still, there is a general lack of standards for implementing a Blockchain approach for traceability. A Blockchain must be universal and adaptable to specific situations \cite{valenta2017comparison}. In addition, the need to agree on a particular type of Blockchain to be used puts the parties under pressure. 

Compared to traditional systems, the great difference of this work is to provide the non-functional requirements acquired by using the blockchain:

\begin{itemize}
\item More reliable operations;
\item More democratic transactions;
\item Process optimization;
\item Ease of coordination between companies;
\item Recording data in chronological order;
\item Cost reduction;
\item Distributed and autonomous platform.
\end{itemize}

In relation to blockchain-based systems:

\begin{itemize}
\item Not specific to a product type;
\item Open source project and not closed;
\end{itemize}

Our work is intended to provide a Blockchain-based platform to facilitate the development of applications for traceability in supply chain management.

\begin{table}[H]
\caption{Related Works and their main characteristics, strategies and results.}
\label{table:userStories}
    \begin{tabular}{|l|p{2.1cm}|p{3.2cm}|p{4.5cm}|}
    \hline 
    \thead{Related work} & \thead{Blockchain} & \thead{Solution} & \thead{Main Deficiency} \\ 
    \hline 
    Microsoft Dynamics 365  & No & Simple SCM needs & Centralized solution. Not a distributed and autonomous platform.\\
    \hline
    Plex System  & No & SaaS ERP system & Centralized solution. Not easy to coordinate between companies. High cost. Private System. \\
    \hline 
    Oracle  NetSuite  & No & cloud-based supply chain and ERP system & Centralized solution. High cost. Private System.\\
    \hline 
    SAP SCM  & No & AI and the Internet of Things & High cost. Private System.\\
    \hline 
    Tian  & Yes & Oroposed system that combines HACCP, Blockchain, and IoT & Theoretical application. Food related only.\\
    \hline 
    Everledger Diamonds  & Yes & Blockchain-based solution & Product specific. Not an open source project. High cost. \\
    \hline 
    IBM Food Trust  & Yes & Blockchain-based solution & Product specific. Not an open source project. High cost.\\
    \hline
    SCM-BP  & Yes & Blockchain-based solution & Relies on a minimum amount of participation to obtain a reliable forecast.\\
    \hline
    \end{tabular}
\end{table}
%\input{chapters/3_GENERAL_CONTEXT/3_GENERAL_CONTEXT.tex}
\xchapter{Proposed Solution}{} %sem preambulo
\label{chap:Technical}

\acresetall 

\ac{SCM-BP} has the general objective to create a generic framework intended to be used in any supply chain correlated to assets and products. 

An SCM platform relies on three main items: assets (the goods themselves), steps (phases which products go through), and actors (people who transact assets during the steps). Our approach is based on this triad that must be defined to create a new supply chain. The workflow is divided into two phases: design time and execution time. Design time is when an administrator configures the asset flow: define asset info, steps, actors and create the relationship between these entities. Execution time is composed of creating, moving, and tracking asset items through the supply chain.

In the design time, initially, a configuration file in JSON format is generated and read in the Blockchain platform, adding the primary information for the correct functioning of the chain. The mechanism for creating this configuration file is detailed in \Cref{sec:UserInteraction,sec:ServiceLayer}.


\section{Application Architecture}\label{sec:applicationArchitecture}

The main objective of this work is to create a generic platform for Supply Chain Management (SCM). \ac{SCM-BP} is divided into four main modules described below: WebApp - frontend, WebApp - backend, Blockchain network and Data Storage. Figure~\ref{fig:detalhamentotecnico} shows the application architecture and its components. Figure~\ref{fig:dataStructure} present the main data structure.

\begin{figure}[htbp]
\begin{center}
  \includegraphics[scale=0.55]{images/architecture.png}
\caption{Application architecture of \ac{SCM-BP}}
\label{fig:detalhamentotecnico}
\end{center}
\end{figure}

In the table~\ref{table:userStories} are defined the main functional requirements in order to be able to meet all the objectives proposed by this project. Are characterized as non-functional requirements for the proper functioning the items in table~\ref{table:rnf}.



\subsection{WebApp - FrondEnd}\label{sec:WebAppFrondEnd}
WebApp - FrontEnd is a client-server computer application that the client (including the user interface and client-side logic) runs in a web browser. This is a \ac{SPA}, a web application that interacts with the user by dynamically rewriting the current page rather than loading entire new pages from a server. This approach avoids interruption of the user experience between successive pages, making the application behave more like a desktop application.

The application is built with Angular, a JavaScript library for building user interfaces. It is a TypeScript-based open-source web application framework led by the Angular Team at Google and by a community of individuals and corporations. Used as a base in developing of single-page or mobile applications, Angular is optimal for fetching rapidly changing data that needs to be recorded. However, fetching data is only the beginning of what happens on a web page, which is why complex Angular applications usually require additional libraries for state management, routing, and interaction with an API.

The Webapp - FrontEnd is divided into two main blocks, and these are classified according to the interactions: User Interaction Modules and Backend Interactions Services.

\subsubsection{User Interaction}\label{sec:UserInteraction}
The User Interaction modules are responsible for providing web pages that will be rendered on the client's web browser. These interactions are provided by web pages grouped by the following components:

\begin{itemize}
\item Login page
\item Application configuration module
\item User handling module (actors - CRUD)
\item Data entry module (forms)
\item Data visualization module
\item Reporting module
\end{itemize}

The Login Module is responsible for display the login and authentication alternatives pages (e.g. ‘forgot my password’, ‘reset my password’). The Application Configuration module provides the features of the creation/configuration of supply chain items and supply chain flows (steps). This module is responsible for getting the information from the user to generate the configuration JSON file in the backend. The User handling module provides the features for the creation/configuration of Actors and Steps, complementing the configuration file. The Data Entry module provides form pages that allow the actors to enter data in the application, search and move asset items from a step to another. The Data Visualization module is responsible for displaying the information about asset items in the supply chain flow through steps. In the Reporting module, users can generate reports/files organized in a narrative, graphic, or tabular form, prepared on ad hoc, periodic, recurring, regular, or as required. Reports may refer to specific periods, events, occurrences, or subjects presented in written form or any other format.


\subsubsection{Backend Interaction}\label{sec:BackendInteraction}
Backend interactions happen via a service layer consisting of:

\begin{itemize}
\item Authentication service
\item Application setup service
\item User creation service (actors)
\item Data entry service (forms)
\item Data visualization service
\item Reporting service
\end{itemize}

The function of the Authentication Service is to request information from an authenticating party and validate it against the configured identity repository using the specified authentication module. After successful authentication, the user session is activated and validated across all web applications participating in an SSO environment. For example, when a user or application attempts to access a protected resource, credentials are requested by one (or more) authentication modules. Gaining access to the resource requires that the user or application be allowed based on the submitted credentials.

Application setup service provides methods to configure and edit supply chain items, and supply chain flows, defining which steps and sub-tasks will be present in this flow and which information will be present in these steps.

The User creation Service is responsible for creating users and roles to log in and use the application’s features. Only Administrators are allowed to create new users (see Actions and Actors).

Data entry service receives data from UI forms and sends them to the backend to be processed and stored.

Data visualization services provide information about the supply chain: Assets, users, and transactions, to be used by the data visualization module.

Report services generate files (Doc/PDF/XSL, etc...) from a specific period with information about the supply chain: Assets, users, and transactions.
\subsection{WebApp - BackEnd}\label{sec:WebAppBackEnd}
WebApp - BackEnd is a Middleware that runs on the server. This Middleware (server-side software) facilitates client-server connectivity, forming a middle layer between the app(s) and the network: the server, the database, the operating system, and more. It receives requests from the clients (in this case, the WebApp - FrontEnd), and contains the logic to send the appropriate data back to the applicant, over HTTP and REST.  These are the main conventions that provide structure to the request-response cycle between clients and servers.

WebApp - BackEnd is an application build with Node.js, an application platform where developers can write Javascript programs that are compiled, optimized and interpreted by the V8 virtual machine. Node.js can create quick, reliable websites and products in much efficient manner. Developing easy to scale real time applications in other technologies is bit difficult, but JavaScript technologies made it easier.

The WebApp - BackEnd is composed by the API Gateway, Service Layer and Resource Locator more detailed below.

\subsubsection{API Gateway}\label{sec:APIGateway}
API Gateway is a managed service that enables easily create, publish, maintain, monitor and secure REST APIs to act as a "gateway" for applications to access data, business logic, or functionality in the backend services, such as workloads. The API Gateway provides a simple uniform view of external resources to the internals of this application. It manages all tasks involved in receiving and processing API calls, including traffic management, authorization and access control, monitoring and management of API versions.

\begin{figure}[htbp]
\begin{center}
  \includegraphics[scale=0.75]{images/apigateway.png}
\caption{API Gateway.}
\label{default-regular2}
\end{center}
\end{figure}

Basically, the Gateway is an interface that receives calls to its internal systems, being a large gateway. It act in five different ways:

\begin{itemize}
\item Filter for call traffic from different media;
\item Single gateway to the various APIs that are exposed;
\item Router: API and Rate Limit traffic router;
\item Security engine with authentication and logging.
\end{itemize}

Gateway access can be done from many different devices. Therefore, it must have the power to unify outgoing calls and be able to deliver to the user content that can be accessed from any browser and system. In this project the gateway interaction happens with the frontend webapp. The Gateways as a Security Feature: In the APIs world, one of the most subject talked about issues is always security, and having an API Gateway is one of the best solutions on the market to get full control of API’s, because this pattern addresses the so-called CIA (Confidentiality, Integrity, Availability) almost flawlessly.

\subsubsection{Service Layer}\label{sec:ServiceLayer}
A Service Layer defines an application's boundary and its set of available operations from the perspective of interfacing client layers. It encapsulates the application's business logic, controlling transactions and coordinating responses in the implementation of its operations.

Enterprise applications typically require different kinds of interfaces to the data they store and the logic they implement: data loaders, user interfaces, integration gateways, and others. Despite their different purposes, these interfaces often need common interactions with the application to access and manipulate its data and invoke its business logic. The interactions may be complex, involving transactions across multiple resources and the coordination of several responses to an action. Encoding the logic of the interactions separately in each interface causes a lot of duplication. the service layer provides:

\begin{enumerate}
\item Centralizes external access to data and functions.
\item Hides (abstracts) internal implementation and changes.
\item Allows for versioning of the services.
\end{enumerate}

The service layer acts as an orchestrator, controlling the flow of incoming and outcoming information requests and responses. Orchestration allows to directly link process logic to service interaction within workflow logic. This combines business process modeling with service-oriented modeling and design, realizing workflow management through a process service model. Orchestration brings the business process into the service layer, positioning it as a master composition controller.

\subsubsection{Resource Locator}\label{sec:ResourceLocator}

Resource locators are components that abstracts the persistence layer. Their job is to provide an object that can help services to discover and persist information from/to the Data Storage Module. Information can be stored in the Blockchain, Filesystem or Database and resource locators should know exactly where get/put data within them.  
\subsection{Blockchain}\label{sec:BlockchainModule}

A blockchain is a peer-to-peer distributed ledger forged by consensus, combined with a system for “smart contracts” and other assistive technologies. Together these can be used to build a new generation of transactional applications that establishes trust, accountability and transparency at their core, while streamlining business processes and legal constraints.

SCM-BP uses Blockchain as a supply chain that track parts and service provenance, ensure authenticity of goods, block counterfeits and reduce conflicts.

To achieve that, Hyperledger Fabric is used. Hyperledger is an open source collaborative effort created to advance cross-industry blockchain technologies. It is a global collaboration, hosted by The Linux Foundation, including leaders in finance, banking, Internet of Things, supply chains, manufacturing and Technology.
Hyperledger Fabric is an enterprise-grade permissioned distributed ledger framework for developing solutions and applications. Its modular and versatile design satisfies a broad range of industry use cases. It offers a unique approach to consensus that enables performance at scale while preserving privacy.

In context of SCM-BP, the Blockchain module consists in a smart contract, chaincode and the ledger. From the application developer’s perspective, a smart contract, together with the ledger, form the heart of a Hyperledger Fabric blockchain system. Whereas a ledger holds facts about the current and historical state of a set of business objects, a smart contract defines the executable logic that generates new facts that are added to the ledger. A chaincode is typically used by administrators to group related smart contracts for deployment, but can also be used for low level system programming of Fabric.

\subsubsection{Smart contract}
Before businesses can transact with each other, they must define a common set of contracts covering common terms, data, rules, concept definitions, and processes. Taken together, these contracts lay out the business model that govern all of the interactions between transacting parties.

A smart contract defines the rules between different organizations in executable code. Applications invoke a smart contract to generate transactions that are recorded on the ledger.

\subsubsection{Chaincode}
Hyperledger Fabric users often use the terms smart contract and chaincode interchangeably. In general, a smart contract defines the transaction logic that controls the lifecycle of a business object contained in the world state. It is then packaged into a chaincode which is then deployed to a blockchain network. Think of smart contracts as governing transactions, whereas chaincode governs how smart contracts are packaged for deployment.

\subsubsection{Ledger}
At the simplest level, a blockchain immutably records transactions which update states in a ledger. A smart contract programmatically accesses two distinct pieces of the ledger – a blockchain, which immutably records the history of all transactions, and a world state that holds a cache of the current value of these states, as it’s the current value of an object that is usually required.

Smart contracts primarily put, get and delete states in the world state, and can also query the immutable blockchain record of transactions.

\begin{itemize}
\item A \textbf{get} typically represents a query to retrieve information about the current state of a business object.
\item A \textbf{put} typically creates a new business object or modifies an existing one in the ledger world state.
\item A \textbf{delete} typically represents the removal of a business object from the current state of the ledger, but not its history.
\end{itemize}

Smart contracts have many APIs available to them. Critically, in all cases, whether transactions create, read, update or delete business objects in the world state, the blockchain contains an immutable record of these changes.

\subsubsection{Implementation Details}\label{sec:Implementation}

The chaincode is written in Golang and provides all contracts needed to proceed traceability in the application. All contracts for use in chaincode must implement the interface \textit{contractapi.ContractInterface}. 

The first step is to create a JSON config file providing all information about these three items. A configuration file includes \textit{assetId}, a list of actors and a list of ordered steps. The chaincode processes this file through  \textit{initLedger} and \textit{createNewAsset} functions. Here follows a template for config file:  

% \begin{minted}{go}
% func main() {
%   chaincode, err := contractapi.NewChaincode(
%     new(SmartContract)
%   )
%   if err != nil {
%     fmt.Printf("Error create chaincode: %s", err.Error())
%     return
%   }
%   if err := chaincode.Start(); err != nil {
%     fmt.Printf("Error starting chaincode: %s", err.Error())
%   }
% }
% \end{minted}

\begin{minted}{json}
{
   "AssetId":"assetName",
   "Actors":[
      {
         "actorType":"type",
         "aditionalInfo":[
            {
               "key":"value"
            }
         ]
      }
   ],
   "Steps":[
      {
         "step":"stepName",
         "stepOrder":1,
         "actorType":"actorType"
      }
   ]
}
\end{minted}

Front-end WebApp enables a user to define settings through a Configuration Page, adding these to the configuration file, as shown in Figure~\ref{fig:frontend02}.
\textcolor{red}{TODO: ATUALIZAR IMAGEM NA NOVA SEÇÃO, ATUALIZAR REFERENCIA}

%htbp
\begin{figure}[ht]
\begin{center}
  \includegraphics[scale=0.265]{images/frontend02.png}
\caption{SCM Configuration page}
\label{fig:frontend02}
\end{center}
\end{figure}

Assets, asset items, steps and actors are described as \textit{structs}, as follows:

\begin{minted}{go}
type Actor struct {
  ActorId    string  `json:"actorId"`
  ActorType  string  `json:"actorType"`
  ActorName  string  `json:"actorName"`
  Deleted    bool    `json:"deleted"`
  AditionalInfo map[string]string `json:"  aditionalInfo"`
}

type Step struct {
  StepId     string  `json:"stepId"`
  StepName   string  `json:"stepName"`
  StepOrder  uint    `json:"stepOrder"`
  ActorType  string  `json:"actorType"`
  Deleted    bool    `json:"deleted"`
  AditionalInfo map[string]string `json:"  aditionalInfo"`
}

type AssetItem struct {
  AssetItemId   string    `json:"assetItemId"`
  OwnerId       string    `json:"ownerId"`
  StepID        string    `json:"stepID"`
  ParentID      string    `json:"parentID"`
  Children      []string  `json:"children"`;
  ProcessDate   string    `json:"processDate"`
  DeliveryDate  string    `json:"deliveryDate"`
  OrderPrice    string    `json:"orderPrice"`
  ShippingPrice string    `json:"shippingPrice"`
  Status        string    `json:"status"`
  Quantity      string    `json:"quantity"`
  Deleted       bool      `json:"deleted"`
  AditionalInfo map[string]string `json:"  aditionalInfo"`
}

type Asset struct {
  AssetId      string       `json:"assetId"`
  AssetName    string       `json:"assetName"`
  AssetItems   []AssetItem  `json:"assetItems"`
  Actors       []Actor      `json:"actors"`
  Steps        []Step       `json:"steps"`
  Deleted      bool         `json:"deleted"`
  AditionalInfo map[string]string `json:"aditionalInfo"`
}
\end{minted}

There are create methods for each one,  responsible for creating an instance of these \textit{structs} and save the state into the Blockchain. Query methods are responsible for interact with the information of any item in the Blockchain.


The \textit{main} function of chaincode invokes the \textit{initLedger} function, reads the configuration files and raises the platform enabling users to interact with the Blockchain via exposing its API, as follows: 

\begin{minted}{go}
func main() {
  chaincode, err := contractapi.NewChaincode(new(SmartContract))
  if err != nil {
    fmt.Printf("Error create chaincode: %s", err.Error())
    return
  }
  if err := chaincode.Start(); err != nil {
    fmt.Printf("Error starting chaincode: %s", err.Error())
  }
}
\end{minted}

When creating an asset item, an \textit{AssetItemId} is generated. Each entity in the chain will have its unique entity ID and timestamp when it starts processing the transaction. By querying \textit{AssetItemId}, the user can easily track the current transaction information and status. Finally, completed all steps, the Blockchain will update \textit{deliverDate} and mark the status as completed once the last actor (generally the consumer) has received the order. Here follows \textit{CreateAsset} function:

\begin{minted}{go}
func (s *SmartContract) CreateAsset(
  ctx contractapi.TransactionContextInterface, assetId string,
  assetName string, assetItems []AssetItem, actors []Actor,
  steps []Step, aditionalInfo map[string]string) error {
  
  if err != nil {
    return fmt.Errorf(
      "Failed to read the data from world state: %s", err
    )
  }

  if assetJSON != nil {
    return fmt.Errorf("The asset %s already exists", assetID)
  }
  
  asset := Asset {
    AssetId:       assetId,
    AssetName:     assetName,
    AssetItems:    assetItems,
    Actors:        actors,
    Steps:         steps,
    Deleted:       false,
    aditionalInfo: aditionalInfo,
  }
  assetAsBytes, _ := json.Marshal(asset)
  if err != nil {
    return err
  }
  return ctx.GetStub().PutState("ASSET"+assetId,assetAsBytes)
}
\end{minted}

\textit{MoveAssetItem} is the method called to update an asset item when it is moved from a step to another. It updates the \textit{CurrentOwnerId}, the \textit{ProcessDate}, information about prices and many other details of the transactions by the key/value map \textit{  aditionalInfo}. Here follows this fuction:

\begin{minted}{go}
func (s *SmartContract) MoveAssetItem(
  ctx contractapi.TransactionContextInterface, 
  assetItemID string, newAssetItemID string, stepID string, 
  newOwnerID string, orderPrice string, shippingPrice string, 
  status string, quantity string, 
  aditionalInfo map[string]string ) error {
  
  _, err := s.QueryAssetItem(ctx, assetItemID)
  if err != nil {
    return err
  }

  newAssetItem := AssetItem{
    AssetItemID:      newAssetItemID,
    OwnerID:          newOwnerID,
    StepID:           stepID,
    ParentID:         assetItemID,
    ProcessDate:      time.Now().Format("2006-01-02 15:04:05"),
    OrderPrice:       orderPrice,
    ShippingPrice:    shippingPrice,
    Status:           status,
    Quantity:         quantity,
    Deleted:          false,
    AditionalInfoMap: aditionalInfo,
  }

  assetItemAsBytes, err := json.Marshal(newAssetItem)
  if err != nil {
    return err
  }

  return ctx.GetStub().PutState(
    "ASSET_ITEM_"+newAssetItemID, assetItemAsBytes
  )
}
\end{minted}

\textcolor{red}{TODO: UPDATE TRACK-ASSET-ITEM METHOD}

\textit{TrackAssetItem} is the method responsible for track an asset item. It returns the children tree of the given element and its ancestors from the beginning of the Supply chain.

\begin{minted}{go}
func (s *SmartContract) TrackAssetItem(
  ctx contractapi.TransactionContextInterface, 
  assetItemID string) ([]*AssetItem, error) {
  
  assetItem, err := s.QueryAssetItem(ctx, assetItemID)
  log.Print("tracking assetItem id: ", assetItem.AssetItemID)
  if err != nil {
    return nil, fmt.Errorf("Failed from world state. %s", err.Error())
  }

  if assetItem == nil {
    return nil, fmt.Errorf("%s does not exist", assetItemID)
  }

  trackedItems := make([]*AssetItem, 0)
  trackedItems = append(trackedItems, assetItem)
  for {
    currentParentId, err := strconv.Atoi(assetItem.ParentID)
    log.Print("currentParentId: ", currentParentId)
    if currentParentId <= 0 {
      log.Print("oldParentId is equals or less than 0. break it")
    break
    }
    parentAssetItem, err := s.QueryAssetItem(ctx, assetItem.ParentID)
    if err != nil {
      return nil, fmt.Errorf(
        "Failed to read from world state. %s", err.Error()
      )
    }
    newParentId, err := strconv.Atoi(parentAssetItem.ParentID)
    log.Print("newParentId: ", newParentId)
    trackedItems = append(trackedItems, parentAssetItem)
    assetItem = parentAssetItem
  }
  return trackedItems, nil
}
\end{minted}


Figure~\ref{fig:sequenceDiagram} shows the interaction flow from users with Árion platform. Initially, an admin persona creates and configure the SCM, adding information about the steps and the users. After that, the admin can activate this SCM, and from that point, the actors can interact with the SCM to provide information about an asset item and also move this asset item through the supply chain. From that point too, any user can track an asset item to get information about the required good.

%htbp
\begin{figure}[ht]
\begin{center}
  \includegraphics[scale=0.5]{images/SequenceDiagram.png}
\caption{SCM User flow}
\label{fig:sequenceDiagram}
\end{center}
\end{figure}

\subsection{Data Storage}\label{sec:DataStorage}
Data storage is a general term for archiving data in electromagnetic or other forms for use by a computer or device. Different types of data storage play different roles in a computing environment. In addition to forms of hard data storage, there are now new options for remote data storage, such as cloud computing, and blockchain that can revolutionize the ways that users save and access data.  

SCM-BP uses three applications as data storages: Blockchain, Cloud filesystem and relational database better detailed on next subsections. Blockchains grow continuously because of the amount of data and code in them, which is unchanging. Therefore, an important design decision is to choose which data and calculations to keep in and out of the chain.

\subsubsection{Filesystem}\label{sec:Filesystem}
A cloud file system is a tiered storage system that provides shared access to file data. Users can create, delete, modify, read and write files, as well as logically organize them into directory trees for intuitive access.

Cloud file sharing can be defined as a service that gives multiple users simultaneous access to a cloud file data set. Cloud file sharing security is managed with user and group permissions, allowing administrators to tightly control access to shared file data.

For all file uploaded and stored in the filesystem, a locally stored digital fingerprint (hash) is saved in the blockchain, separately from the original files or content, to make it easier to confirm whether data has been altered or manipulated in a particular organization.

\subsubsection{Database}\label{sec:Database}
A relational database is a set of formally described tables from which data can be accessed or reassembled in many different ways without having to reorganize the database tables. The standard user and application programming interface (API) of a relational database is the Structured Query Language (SQL). SQL statements are used both for interactive queries for information from a relational database and for gathering data for reports.

\subsubsection{Blockchain}\label{sec:DataStorageBlockchain}

Since the blockchain consists of a Ledger and a world state database (among another components), this can also be seen as part of data storage module as it stores data. The \ref{sec:ResourceLocator} has the intelligence to decide where to search and store data with the objective of optimize storage consumption.
\section{Actions And Actors}\label{sec:actionsAndActors}

The system is governed by a set of rules. These rules define how users are to interact with the system, and how the data is shared among the users. Moreover, once the rules are stored in the blockchain, they can not be altered without broadcasting to all nodes and verified by most of them.
%%% SETUP

\subsection{Setup}\label{sec:Setup}
Setup is the set of actions to configure the application. Setup phase is when a new supply chain is created or an existing one is updated or deleted. Users can also be created, updated and deleted by setup actions. When creating or editing a supply chain, Admin users will define which steps, sub-steps and information that the supply chain flow will contain, and which users from member group is allowed to add info and move asset to each step in the logistics network.

\subsubsection{Administrator}\label{sec:Administrator}
Administrator is the only user in Admin group. This actor has access to all areas of the program and have the same abilities that all other user types (configure, move asset and view flow), but his main responsibility is to configure the application, performing the setup actions. 


%%% DATA INSERTION

\subsection{Data Insertion}\label{sec:DataInsertion}

Data insertion are the actions that will fill the supply chain flow with data. Once a new supply chain is created by the admin user, it is ready to be populated with information. Member and admin users are responsible for perform these actions. In Data Insertion phase users can update information from a specific step and sub-step and move assets depending on the rules applied in the setup phase. The most common actor types in SCM are: Raw material/Producer, Manufacturer, Distributor Wholesaler and Retailer.

%%% VISUALIZATION

\subsection{Visualization}\label{sec:Visualization}

Visualization actions can be performed by any user in the system but its main purpose is to provide to the end user the capability of track the flow off an asset from point of origin to point of consumption.
\section{Implementation Details}\label{sec:Implementation}

The chaincode is written in Golang and provides all contracts needed to proceed traceability in the application. All contracts for use in chaincode must implement the interface \textit{contractapi.ContractInterface}. 

The first step is to create a JSON config file providing all information about these three items. A configuration file includes \textit{assetId}, a list of actors and a list of ordered steps. The chaincode processes this file through  \textit{initLedger} and \textit{createNewAsset} functions. A template for config file can be found in the appendix \ref{app:template}.

Front-end WebApp enables a user to define settings through a Configuration Page, adding these to the configuration file, as shown in Figure~\ref{fig:create_asset_4}. Assets, asset items, steps and actors are described in the appendix \ref{app:structs}. There are create methods for each one,  responsible for creating an instance of these \textit{structs} and save the state into the Blockchain. Query methods are responsible for interact with the information of any item in the Blockchain.

The chaincode \textit{main} function invokes the \textit{initLedger} function, reads the configuration files and raises the platform enabling users to interact with the Blockchain via exposing its API.

When creating an asset item, an \textit{AssetItemId} is generated. Each entity in the chain will have its unique entity ID and timestamp when it starts processing the transaction. By querying \textit{AssetItemId}, the user can easily track the current transaction information and status. Finally, completed all steps, the Blockchain will update \textit{deliverDate} and mark the status as completed once the last actor (generally the consumer) has received the order. The \textit{CreateAsset} function is detailed on the appendix \ref{app:CreateAsset}.


\textit{MoveAssetItem} is the method called to update an asset item when it is moved from a step to another. It updates the \textit{CurrentOwnerId}, the \textit{ProcessDate}, information about prices and many other details of the transactions by the key/value map \textit{aditionalInfo}. Its details is on the appendix \ref{app:MoveAssetItem}.

\textit{TrackAssetItem} is the method responsible for track an asset item. It returns the children tree of the given element and its ancestors from the beginning of the Supply chain. This method uses the function getChildenTree to get the current item's children nodes. The appendix \ref{app:TrackAssetItem} contains its implementation.

For audit purposes some methods where implemented in order to get info about the blocks and transactions stored in the ledger. These are the methods: \textit{getBlockByTxID, getChainInfo, getBlockByHash, getBlockByNumber,} and \textit{getTransactionByID}. The appendix \ref{app:auditMethods} contains their implementation.

Figure~\ref{fig:sequenceDiagram} shows the interaction flow from users with Árion platform. Initially, an admin persona creates and configure the SCM, adding information about the steps and the users. After that, the admin can activate this SCM, and from that point, the actors can interact with the SCM to provide information about an asset item and also move this asset item through the supply chain. From that point too, any user can track an asset item to get information about the required good.

%htbp
\begin{figure}[ht]
\begin{center}
  \includegraphics[scale=0.5]{images/SequenceDiagram.png}
\caption{SCM User flow}
\label{fig:sequenceDiagram}
\end{center}
\end{figure}


The backend gateway exposes all the endpoints shown in table \ref{table:endpoints}. These endpoints are integrated with the frontend and can be used in the future work to integrate with a mobile app or an external application.


\begin{table}[htb]
\centering
\caption{Backend endpoints}
\label{table:endpoints}
    \begin{center}
    \begin{tabular}{|l|p{4.5cm}|p{7.664cm}|}
        \hline 
        \thead{Verb} & \thead{URI} & \thead{Description}\\
        \hline 
        \cellcolor{cyan}\textbf{Actors}  & \cellcolor{cyan}\textbf{} & \cellcolor{cyan}\textbf{} \\
        \hline 
        POST & /actors & creates a new actor\\
        \hline 
        GET & /actors & retrieves the actors' list\\
        \hline 
        PUT & /actors/:id & updates an existing actor\\
        \hline 
        GET & /actors/:id & retrieves an existing actor\\
        \hline 
        DELETE & /actors/:id & removes an existing actor\\
        \hline 
        \cellcolor{cyan}\textbf{Steps}  & \cellcolor{cyan}\textbf{} & \cellcolor{cyan}\textbf{} \\
        \hline 
        POST & /steps & creates a new step\\
        \hline 
        GET & /steps & retrieves the steps' list\\
        \hline 
        PUT & /steps/:id & updates an existing step\\
        \hline 
        GET & /steps/:id & retrieves an existing step\\
        \hline 
        DELETE & /steps/:id & removes an existing step\\
        \hline 
        \cellcolor{cyan}\textbf{Asset Items}  & \cellcolor{cyan}\textbf{} & \cellcolor{cyan}\textbf{} \\
        \hline 
        POST & /asset-items & creates a new asset item\\
        \hline 
        GET & /asset-items & retrieves the asset items' list\\
        \hline 
        PUT & /asset-items/:id & updates an existing asset item\\
        \hline 
        GET & /asset-items/:id & retrieves an existing asset item\\
        \hline 
        DELETE & /asset-items/:id & removes an existing asset item\\
        \hline 
        POST & /asset-items/:id & moves an asset item through the SCM\\
        \hline 
        GET & /asset-items/track/:id & tracks an asset item through the SCM\\
        \hline
        \cellcolor{cyan}\textbf{Assets}  & \cellcolor{cyan}\textbf{} & \cellcolor{cyan}\textbf{} \\
        \hline 
        POST & /assets & creates a new asset\\
        \hline 
        GET & /assets & retrieves the assets' list\\
        \hline 
        PUT & /assets/:id & updates an existing asset\\
        \hline 
        GET & /assets/:id & retrieves an existing asset\\
        \hline 
        DELETE & /assets/:id & removes an existing asset\\
        \hline 
        \cellcolor{cyan}\textbf{Audit}  & \cellcolor{cyan}\textbf{} & \cellcolor{cyan}\textbf{} \\
        \hline
         GET & /blocks/:number & Return the block specified by block number\\
        \hline 
         GET & /blocks/hash/:hash & Return the block specified by block hash\\
        \hline 
         GET & /blocks/tx-id/:tx-id & Return the transaction specified by Transaction ID\\
        \hline 
         GET & /transactions:id & Return the transaction specified by ID\\
        \hline 
         GET & /chain & Return a Blockchain Info object marshalled in bytes\\
        \hline 
    \end{tabular}
    \end{center}
\end{table}
%\input{chapters/4_TECHNICAL_SPECIFICATION/sections/3_stepsSubsteps.tex}

%\input{chapters/4_TECHNICAL_SPECIFICATION/sections/3_targetGoals.tex}
%\input{chapters/4_TECHNICAL_SPECIFICATION/sections/4_methods.tex}
\xchapter{Use Example}{} %sem preambulo
\label{chap:useExample}

\acresetall 

\section{Scientific Method}\label{sec:method}

For this project development, the agile Scrum method has been used. In Scrum, projects are divided into cycles called sprints, with frequent meetings where the team can inform what is being done and think of ways to quickly improve the process. Scrum proposes constant project monitoring. Often the team will be meeting, exchanging experiences, evaluating what has been done and re-planning what will be done next.

During the requirements gathering, developers and other stakeholders sought to raise and prioritize the needs of future software users (referred as requirements). After the requirements gathering, in the requirements specification stage, developers made a detailed study of data collected in the previous activity, from where models were built to represent the software system being developed.

At the architectural design stage of the system two basic activities were performed: architectural design (or high level design), and detailed design (or low level design). Some aspects were considered at this stage of system design, such as: system architecture, platform used, Database Manager System (DBMS) used and graphical interface standard.

In the application development period, the backend and frontend components were created from the computational description of the design phase. Pre-existing software tools and class libraries were used to activity streamline. These tools and libraries were defined during the architectural design and were referenced in \cref{sec:WebAppFrondEnd,sec:WebAppBackEnd} .

For system validation, two main requirements were evaluated: the components and the behavior of who will use the application. For the first point, functional, integration and security tests will be performed. For the second, the \acf{TAM} method was used to evaluate user acceptance, utility and ease of use.

The appendix \ref{app:projectManagement} presents the activities for project management, the user stories and non-functional requirements. 


\section{Workflow}\label{sec:workflow}

The first step when creating a new Supply Chain  is to create and configure the asset. This action is made by the admin user. By going through the setup wizard, first the asset's info is requested:

\begin{figure}[H]
\begin{center}
  \includegraphics[scale=0.27]{images/use_example/01_create_asset_1.png}
\caption{Fill asset info.}
\label{fig:create_asset_1}
\end{center}
\end{figure}

Then the admin will add actors to the SCM informing its types:

\begin{figure}[H]
\begin{center}
  \includegraphics[scale=0.27]{images/use_example/02_create_asset_2.png}
\caption{Adding actors.}
\label{fig:create_asset_2}
\end{center}
\end{figure}


The next step in the wizard is to define the Supply chain steps, specifying the step order and binding it to the previous created actors' types:

\begin{figure}[H]
\begin{center}
  \includegraphics[scale=0.27]{images/use_example/03_create_asset_3.png}
\caption{Defining steps.}
\label{fig:create_asset_3}
\end{center}
\end{figure}

The final step, before submitting it is to review all the information previously added in the review asset details page, under the wizard:
\begin{figure}[H]
\begin{center}
  \includegraphics[scale=0.265]{images/use_example/04_create_asset_4.png}
\caption{Review asset details before submit.}
\label{fig:create_asset_4}
\end{center}
\end{figure}


Once created, the asset can be seen in the assets list, where the admin can perform crud operations by the actions items:
\begin{figure}[H]
\begin{center}
  \includegraphics[scale=0.28]{images/use_example/05_asset_list.png}
\caption{Asset list and action buttons.}
\label{fig:asset_list}
\end{center}
\end{figure}

When clicking in the asset details the current user is redirected to the asset details page where the main information about asset items, actors and steps can be seen. The card header contains a tab menu to provide navigation through these entities. 

\begin{figure}[H]
\begin{center}
  \includegraphics[scale=0.35]{images/use_example/06_asset_Item_list.png}
\caption{Asset Items list.}
\label{fig:asset_item_list}
\end{center}
\end{figure}

\begin{figure}[H]
\begin{center}
  \includegraphics[scale=0.34]{images/use_example/07_actor_list.png}
\caption{Actors list.}
\label{fig:actor_list}
\end{center}
\end{figure}

\begin{figure}[H]
\begin{center}
  \includegraphics[scale=0.34]{images/use_example/08_steps_list.png}
\caption{Steps list.}
\label{fig:step_list}
\end{center}
\end{figure}

As admin there is also button actions to perform crud operations such as create an asset item which will redirect the user to the create asset item form: 

\begin{figure}[H]
\begin{center}
  \includegraphics[scale=0.34]{images/use_example/09_create_asset_Item.png}
\caption{Create asset item form.}
\label{fig:create_asset_item}
\end{center}
\end{figure}


In the asset items list, beyond the crud actions there is also two new actions: move asset item and track asset item. The first one shows a form where the user can move an asset through the SCM steps. The user can only move this asset item to the next or the previous step in the supply chain.

\begin{figure}[H]
\begin{center}
  \includegraphics[scale=0.34]{images/use_example/091_move_asset_Item.png}
\caption{Move asset item form.}
\label{fig:move_asset_item}
\end{center}
\end{figure}

The second one displays the tracked info about the chosen asset item. It shows the children tree of the selected element and its ancestors.

\begin{figure}[H]
\begin{center}
  \includegraphics[scale=0.275]{images/use_example/092_track_asset_Item.png}
\caption{Track an asset item.}
\label{fig:track_asset_item}
\end{center}
\end{figure}

When clicking in a node in the chart, the information about the selected node is displayed under the diagram.
\xchapter{Conclusion and Future work}{} %sem preambulo
\acresetall 
\label{chap:conclusion}

Lately, Blockchain technology has been the subject of extensive research, but rarely related to supply chain traceability. Although some companies have launched pilot projects using blockchain technology to manage their supply chains, no detailed information on the technical implementation of such projects has been reported. Either way, the retail industry has the potential to use this technology to improve traceability.

Even if some properties of blockchain implementation may be useful for supply chain management, there are still few uses to support this claim. With so little research on this subject, it is difficult for industry stakeholders to understand exactly how blockchain technology can be used in their specific business.

In this paper, was presented a framework for new decentralized traceability systems based on blockchain technology. Moreover, an example scenario was given to demonstrate how it works in a enterprise supply chain. This system deliver real-time information to all supply chain members on the safety status of goods, extremely reduce the risk of centralized information systems, and bring more secure, distributed, transparent, and collaborative to the supply chain management. The Framework can significantly improve development time of Supply Chain Management applications, and provide efficiency and transparency of products in a supply chain.

While joining a blockchain consortium benefits all stakeholders, adopting a new technology such as blockchain is challenging for traditional industries due to the learning curve and cost of integrating blockchain into existing systems. Negotiating business details also takes time. In addition, the development of smart contracts must take into account quality and adaptability. Transparency and data sharing are most important in this regard. Overall, blockchain is a good option for providing traceability in supply chain management. However, the industry needs to look more closely at its risks and opportunities.

As future work, we propose a deeper evaluation that may analyze different product types and accomplish performance tests. Besides, the role permission could be applied to guarantee that only allowed users could read/write sensitive information in the blockchain. This could be made by using a flag in the additional info to show the field as public/private information or better, use the private data collection feature provided by Hyperledger. Also, the asset item's data structure could be changed to a tree data structure for better performance results. 
%% Parte pos-textual
\backmatter

% Bibliografia
\bibliographystyle{abntex2-alf}
\bibliography{biblio}

% Apendices
\appendix

%\lipsum
% Eh aconselhavel criar cada apendice em um arquivo separado, digamos
% "apendice1.tex", "apendice.tex", ... "apendiceM.tex" e depois
% inclui--los com:
% \include{apendice1}
% \include{apendice2}
% ...
% \include{apendiceM}

\xchapter{Project Management}{}\label{app:projectManagement}

\input{apendices/A1_activities}
\section{User stories}{} %sem preambulo

Table \ref{table:userStories} presents all the user stories for artifacts development, used in the system and managed according to agile methodologies.

\begin{table}[H]
\caption{\ac{SCM-BP} User Stories}
\label{table:userStories}
    \begin{tabular}{|l|p{13.5cm}|}
    \hline 
    US-1  & As an administrator, clicking “new” on the supply chain list page takes you to the chain configuration page, with empty settings (no phases, sub-phases, and fields).\\
    \hline 
    US-2  & As an administrator, clicking “edit” on the supply chain list page takes you to the chain configuration page, with the settings filled in (with phases, sub-phases, and fields already registered). \\
    \hline
    US-3  & As an administrator, when clicking delete on the supply chain list page, a modal should appear requesting deletion confirmation.\\
    \hline
    US-4  & As an administrator, when clicking  confirm deletion on the supply chain list page, an alert should appear stating the deletion result: alert-success or alert-danger.\\
    \hline
    US-5  & As an administrator, on the creation or editing screens of a chain, the administrator must tell from each section which user types can enter information in that section.\\
    \hline
    US-6  & As an administrator, when clicking on the User List page redirects to the user creation page with its empty settings.\\
    \hline
    US-7  & As an administrator, on the user creation page, the admin have to enter the type of user (Admin, Producer, Manufacturer, Distributor, Wholesaler, Retailer, End User).\\
    \hline
    US-8  & As an administrator, clicking “edit” on the User List page takes you to the user creation page, with the settings filled in (with the previously entered data).\\
    \hline
    US-9  & As an administrator, when clicking delete on the User List page, a modal should appear asking for deletion confirmation.\\
    \hline
    US-10 & As an administrator, when clicking confirm deletion on the User List page, an alert should appear stating the deletion result: alert-success or alert-danger.\\
    \hline
    US-11 & As a "Member" User (Admin, Producer, Manufacturer, Distributor, Wholesaler, Retailer), clicking Move Asset redirects to the information entry page in the chain.\\
    \hline
    US-12 & As a "Member", each user can only enter information regarding the allowed phase in the access rules (e.g., a distributor cannot enter exploration information) as defined in use case 5.\\
    \hline
    US-13 & As any user (Admin, Member, or End User), clicking Track Asset will take you to a page with a list of all assets paged and filtered by date in descending order (most current to oldest).\\
    \hline
    US-14 & As any user (Admin, Member, or End User), by clicking on “Track Asset”, the user can enter an Id in the input search to search.\\
    \hline
    US-15 & As any user (Admin, Member, or End User), by clicking on “Track”, the user will go to a page with all information of the respective asset, from its conception to the current state.\\
    \hline
    \end{tabular}
\end{table}
\input{apendices/A3_nonFunctionalRequirements}
%\input{apendices/A4_userCreationFields}

\xchapter{Smart Contract}{}
\section{Template for config file}{} %sem preambulo
\label{app:template}

\begin{minted}{json}
{
   "AssetId":"assetID",
   "AssetName":"assetName",
   "AssetDescription":"assetDescription",
   "Actors":[
      {
         "actorType":"type",
         "aditionalInfo":[
            {
               "key":"value"
            }
         ]
      }
   ],
   "Steps":[
      {
         "StepId":"stepID",
         "StepName":"stepName",
         "StepOrder":1,
         "ActorType":"actorType",
         "aditionalInfo":[
            {
               "key":"value"
            }
         ]
      }
   ]
}
\end{minted}

\section{Assets, asset items, steps and actors structs}{} %sem preambulo
\label{app:structs}


\begin{minted}{go}
type Actor struct {
  ActorId    string  `json:"actorId"`
  ActorType  string  `json:"actorType"`
  ActorName  string  `json:"actorName"`
  Deleted    bool    `json:"deleted"`
  AditionalInfo map[string]string `json:"  aditionalInfo"`
}

type Step struct {
  StepId     string  `json:"stepId"`
  StepName   string  `json:"stepName"`
  StepOrder  uint    `json:"stepOrder"`
  ActorType  string  `json:"actorType"`
  Deleted    bool    `json:"deleted"`
  AditionalInfo map[string]string `json:"  aditionalInfo"`
}

type AssetItem struct {
  AssetItemId   string    `json:"assetItemId"`
  OwnerId       string    `json:"ownerId"`
  StepID        string    `json:"stepID"`
  ParentID      string    `json:"parentID"`
  Children      []string  `json:"children"`;
  ProcessDate   string    `json:"processDate"`
  DeliveryDate  string    `json:"deliveryDate"`
  OrderPrice    string    `json:"orderPrice"`
  ShippingPrice string    `json:"shippingPrice"`
  Status        string    `json:"status"`
  Quantity      string    `json:"quantity"`
  Deleted       bool      `json:"deleted"`
  AditionalInfo map[string]string `json:"  aditionalInfo"`
}

type Asset struct {
  AssetId      string       `json:"assetId"`
  AssetName    string       `json:"assetName"`
  Description  string       `json:"description"`
  AssetItems   []AssetItem  `json:"assetItems"`
  Actors       []Actor      `json:"actors"`
  Steps        []Step       `json:"steps"`
  Deleted      bool         `json:"deleted"`
  AditionalInfo map[string]string `json:"aditionalInfo"`
}
\end{minted}

\section{Main function}{} %sem preambulo
\label{app:main}

\begin{minted}{go}
func main() {
  chaincode, err := contractapi.NewChaincode(new(SmartContract))
  if err != nil {
    fmt.Printf("Error create chaincode: %s", err.Error())
    return
  }
  if err := chaincode.Start(); err != nil {
    fmt.Printf("Error starting chaincode: %s", err.Error())
  }
}
\end{minted}

\section{Create Asset}{} %sem preambulo
\label{app:CreateAsset}

\begin{minted}{go}
func (s *SmartContract) CreateAsset(
  ctx contractapi.TransactionContextInterface, assetId string,
  assetName string, description string, assetItems []AssetItem, 
  actors []Actor, steps []Step, aditionalInfo map[string]string) error {
  
  if err != nil {
    return fmt.Errorf(
      "Failed to read the data from world state: %s", err
    )
  }

  if assetJSON != nil {
    return fmt.Errorf("The asset %s already exists", assetID)
  }
  
  asset := Asset {
    AssetId:       assetId,
    AssetName:     assetName,
    Description:   description,
    AssetItems:    assetItems,
    Actors:        actors,
    Steps:         steps,
    Deleted:       false,
    aditionalInfo: aditionalInfo,
  }
  assetAsBytes, _ := json.Marshal(asset)
  if err != nil {
    return err
  }
  return ctx.GetStub().PutState("ASSET_"+assetId,assetAsBytes)
}
\end{minted}

\section{Move asset item}{} %sem preambulo
\label{app:MoveAssetItem}

\begin{minted}{go}
func (s *SmartContract) MoveAssetItem(
  ctx contractapi.TransactionContextInterface, 
  assetItemID string, newAssetItemID string, stepID string, 
  newOwnerID string, orderPrice string, shippingPrice string, 
  status string, quantity string, 
  aditionalInfo map[string]string ) error {
  
  assetItemJSON, err := s.QueryAssetItem(ctx, assetItemID)
  if err != nil {
    return err
  }
  if assetItemJSON == nil {
    return fmt.Errorf("The assetItem %s does not exists", assetItemID)
  }

  newAssetItem := AssetItem{
    AssetItemID:      newAssetItemID,
    OwnerID:          newOwnerID,
    StepID:           stepID,
    ParentID:         assetItemID,
    Children:         []string{},
    ProcessDate:      time.Now().Format("2006-01-02 15:04:05"),
    OrderPrice:       orderPrice,
    ShippingPrice:    shippingPrice,
    Status:           status,
    Quantity:         quantity,
    Deleted:          false,
    AditionalInfoMap: aditionalInfo,
  }

  assetItemAsBytes, err := json.Marshal(newAssetItem)
  if err != nil {
    return err
  }

  return ctx.GetStub().PutState(
    "ASSET_ITEM_"+newAssetItemID, assetItemAsBytes
  )
}
\end{minted}

\section{Track asset item}{} %sem preambulo
\label{app:TrackAssetItem}

\begin{minted}{go}
func (s *AssetTransferSmartContract) TrackAssetItem(
  ctx contractapi.TransactionContextInterface, 
  assetItemID string) ([]*AssetItem, error) {
  
  assetItem, err := s.QueryAssetItem(ctx, assetItemID)
  log.Print("tracking info from assetItem id: ", assetItem.AssetItemID)
  if err != nil {
    return nil, fmt.Errorf(
      "Failed to read from world state. %s", err.Error()
    )
  }

  if assetItem == nil {
    return nil, fmt.Errorf("%s does not exist", assetItemID)
  }

  trackedItems := make([]*AssetItem, 0)

  //first add the children to tracked items
  children, err := s.getChildenTree(ctx, assetItem)
  if err != nil {
    return nil, fmt.Errorf(
      "Failed to read from world state. %s", err.Error()
    )
  }

  for _, child := range children {
    fmt.Println(child)
    trackedItems = append(trackedItems, child)
  }

  //then, add the current item to tracked items
  trackedItems = append(trackedItems, assetItem)

  //finally, add the ancestor of current item to tracked items
  for {
    currentParentId, err := strconv.Atoi(assetItem.ParentID)
    if (currentParentId <= 0) {
      break
    }
    parentAssetItem, err := s.QueryAssetItem(ctx, assetItem.ParentID)
    if err != nil {
      return nil, fmt.Errorf(
        "Failed to read from world state. %s", err.Error()
      )
    }
    newParentId, err := strconv.Atoi(parentAssetItem.ParentID)
    log.Print("newParentId: ", newParentId)
    trackedItems = append(trackedItems, parentAssetItem)
    assetItem = parentAssetItem
  }
  return trackedItems, nil
}

func (s *AssetTransferSmartContract) getChildenTree(
  ctx contractapi.TransactionContextInterface, 
  assetItem *AssetItem) ([]*AssetItem, error) {
  
  tree := make([]*AssetItem, 0)
  log.Print("len(assetItem.Children): ", len(assetItem.Children))
  if len(assetItem.Children) == 0 {
    tree = append(tree, assetItem)
  } else {
    for _, childId := range assetItem.Children {
      childAssetItem, err := s.QueryAssetItem(ctx, childId)
      if err != nil {
        return nil, fmt.Errorf(
          "Failed to read from world state. %s", err.Error()
        )
      }
      if childAssetItem == nil {
        return nil, fmt.Errorf("%s does not exist", childId)
      }

      childrenTree, err := s.getChildenTree(ctx, childAssetItem)
      for _, child := range childrenTree {
        tree = append(tree, child)
      }
    }
    tree = append(tree, assetItem)
  }
  return tree, nil
}
\end{minted}

\section{Audit Methods}{} %sem preambulo
\label{app:auditMethods}

\begin{minted}{go}
func getTransactionByID(
  vledger ledger.PeerLedger, tid []byte) pb.Response {

  if tid == nil {
    return nil, fmt.Errorf("Transaction ID must not be nil.")
  }

  processedTran, err := vledger.GetTransactionByID(string(tid))
  if err != nil {
    return nil, fmt.Errorf(
      "Failed to get transaction with id %s, error %s", 
      string(tid), err.Error()
    )
  }

  bytes, err := protoutil.Marshal(processedTran)
  if err != nil {
    return nil, fmt.Errorf(err.Error())
  }

  return shim.Success(bytes)
}

func getBlockByNumber(
  vledger ledger.PeerLedger, number []byte) pb.Response {
  
  if number == nil {
    return nil, fmt.Errorf("Block number must not be nil.")
  }
  bnum, err := strconv.ParseUint(string(number), 10, 64)
  if err != nil {
    return nil, fmt.Errorf(
      "Failed to parse block number with error %s", err
    )
  }
  block, err := vledger.GetBlockByNumber(bnum)
  if err != nil {
    return nil, fmt.Errorf(
      "Failed to get block number %d, error %s", bnum, err
    )
  }
  
  bytes, err := protoutil.Marshal(block)
  if err != nil {
    return nil, fmt.Errorf(err.Error())
  }

  return shim.Success(bytes)
}

func getBlockByHash(
  vledger ledger.PeerLedger, hash []byte) pb.Response {
  
  if hash == nil {
    return nil, fmt.Errorf("Block hash must not be nil.")
  }
  block, err := vledger.GetBlockByHash(hash)
  if err != nil {
    return nil, fmt.Errorf(
      "Failed to get block hash %s, error %s", string(hash), err
    )
  }

  bytes, err := protoutil.Marshal(block)
  if err != nil {
    return nil, fmt.Errorf(err.Error())
  }

  return shim.Success(bytes)
}

func getChainInfo(vledger ledger.PeerLedger) pb.Response {
  binfo, err := vledger.GetBlockchainInfo()
  if err != nil {
    return nil, fmt.Errorf(
      "Failed to get block info with error %s", err
    )
  }
  bytes, err := protoutil.Marshal(binfo)
  if err != nil {
    return nil, fmt.Errorf(err.Error())
  }
  return shim.Success(bytes)
}

func getBlockByTxID(
  vledger ledger.PeerLedger, rawTxID []byte) pb.Response {
  txID := string(rawTxID)
  block, err := vledger.GetBlockByTxID(txID)
  if err != nil {
    return nil, fmt.Errorf(
      "Failed to get block for txID %s, error %s", txID, err
    )
  }
  bytes, err := protoutil.Marshal(block)
  if err != nil {
    return nil, fmt.Errorf(err.Error())
  }
  return shim.Success(bytes)
}

func (e *LedgerQuerier) Invoke(stub shim.ChaincodeStubInterface) pb.Response {
  args := stub.GetArgs()
  fname := string(args[0])
  cid := string(args[1])
  sp, err := stub.GetSignedProposal()
  name, err := protoutil.InvokedChaincodeName(sp.ProposalBytes)
  targetLedger := e.ledgers.GetLedger(cid)
  qscclogger.Debugf("Invoke function: %s on chain: %s", fname, cid)
  res := getACLResource(fname)

  switch fname {
  case GetTransactionByID:
    return getTransactionByID(targetLedger, args[2])
  case GetBlockByNumber:
    return getBlockByNumber(targetLedger, args[2])
  case GetBlockByHash:
    return getBlockByHash(targetLedger, args[2])
  case GetChainInfo:
    return getChainInfo(targetLedger)
  case GetBlockByTxID:
    return getBlockByTxID(targetLedger, args[2])
  }

  return nil, fmt.Errorf(
    "Requested function %s not found.", fname
  )
}
\end{minted}

\xchapter{Data Structure}{}
\input{apendices/C_dataStructure}

%% Fim do documento
\end{document}
%------------------------------------------------------------------------------------------%
